%!TEX root = ../FYP_Dissertation.tex
\section{Verbose mode}

This module shows verbose information about the progress of the
JIT compiler. It prints one line for each generated trace. This module
is useful to see which code has been compiled or where the compiler
punts and falls back to the interpreter.

Example usage:

\begin{lstlisting}
  luajit -jv -e "for i=1,1000 do for j=1,1000 do end end"
  luajit -jv=myapp.out myapp.lua
\end{lstlisting}
To redirect the output to a file, pass a
filename as an argument (use '-' for stdout) or set the environment
variable LUAJIT\_VERBOSEFILE. The file is overwritten every time the
module is started.

The output from the second example could look like this:

\begin{center}
[TRACE   1 myapp.lua:1 loop]

[TRACE   2 (1/3) myapp.lua:1 $->$ 1]
\end{center}

The first number in each line is the internal trace number. Next are
the file name ('myapp.lua') and the line number (':1') where the
trace has started. Side traces also show the parent trace number and
the exit number where they are attached to in parentheses ('(1/3)').
An arrow at the end shows where the trace links to ('$->$ 1'), unless
it loops to itself.

In this case the inner loop gets hot and is traced first, generating
a root trace. Then the last exit from the 1st trace gets hot, too,
and triggers generation of the 2nd trace. The side trace follows the
path along the outer loop and \textit{around} the inner loop, back to its
start, and then links to the 1st trace.

Aborted traces are shown like this:
\begin{center}
[TRACE --- foo.lua:44 -- leaving loop in root trace at foo:lua:50]
\end{center}

Trace aborts are quite common, even in programs which
can be fully compiled. The compiler may retry several times until it
finds a suitable trace. This doesn't work with features that are
not-yet-implemented (NYI error messages). The VM simply falls back to the
interpreter. This may not matter at all if the particular trace is not very high
up in the CPU usage profile, plus the interpreter is quite fast, too.

\section{Profiler}
\label{Sec:Profiler}

This module is a simple command line interface to the built-in
low-overhead profiler of LuaJIT. The lower-level API of the profiler
is accessible via the "jit.profile" module or the luaJIT\_profile\_* C API.

Example usage:
\begin{lstlisting}
  luajit -jp myapp.lua
  luajit -jp=s myapp.lua
  luajit -jp=-s myapp.lua
  luajit -jp=vl myapp.lua
  luajit -jp=G,profile.txt myapp.lua
\end{lstlisting}
The following dump features are available:
 \begin{itemize}%[label={--}]
  \item \textbf{f} - Shows function name (Default mode).
  \item \textbf{F} - Shows function name with module prepend.
  \item \textbf{l} - Shows line granularity ('module':'line').
  \item \textbf{\textless number\textgreater} - Stack dump depth (callee\textless caller - Default: 1)
  \item \textbf{-\textless number\textgreater} - Inverse stack dump depth (caller\textgreater callee).
  \item \textbf{s} - Split stack dump after first stack level. Implies $\mid depth\mid$
  \textgreater= 2.
  \item \textbf{p} - Show full path for module names.
  \item \textbf{v} - Show VM states (See bellow).
  \item \textbf{z} - Show zones (See bellow).
  \item \textbf{r} - Show raw sample counts (Default: percentages).
  \item \textbf{a} - Annotate excerpts from source code files.
  \item \textbf{A} - Annotate complete source code files.
  \item \textbf{G} - Produce raw output suitable for graphical tools (See bellow).
  \item \textbf{m\textless number\textgreater} - Minimum sample percentage to be shown (Default: 3).
  \item \textbf{i\textless number\textgreater} - Sampling interval in milliseconds (Default: 10).
 \end{itemize}

 Many of those options can be activated at ones.\\

\textbf{\emph{VM states:}}\\
This options allows to shows the time spent in which state of the VM.
States can be of the following types :

\{Compiled - Interpreted - C code - Garbage Collector - JIT Compiler\} \\

\textbf{\emph{Zone:}}\\
statistics can be grouped in user defined zone. Bellow is an example a such a
definition.
\begin{lstlisting}
    local zone = require("jit.zone")
    zone("MyZone")
      -- Lua code here
    zone()
\end{lstlisting}

\textbf{\emph{Graphical tools:}}\\
This option can be used to graphically show the dump in a nice image format
(see Appendix \ref{Apendix:fl})

\section{Dump mode}

This module can be used to debug the JIT compiler itself. It dumps the
code representations and structures used in various compiler stages.

Example usage:
\begin{lstlisting}
  luajit -jdump -e "
    local x=0
    for i=1,1e6 do x=x+i end
    print(x)
  "
  luajit -jdump=im -e "
    for i=1,1000 do
      for j=1,1000 do end
    end
  " | less -R
  luajit -jdump=is myapp.lua | less -R
  luajit -jdump=-b myapp.lua
  luajit -jdump=+aH,myapp.html myapp.lua
  luajit -jdump=ixT,myapp.dump myapp.lua
\end{lstlisting}
The first argument specifies the dump mode. The second argument gives
the output file name. Default output is to stdout, unless the environment
variable LUAJIT\_DUMPFILE is set. The file is overwritten every time the
module is started. Different features can be turned on or off with the dump mode.
If the mode starts with a '+', the following features are added to the default
set of features; a '-' removes them. Otherwise the features are replaced.\\
The following dump features are available (* marks the default):

\begin{itemize}
  \item \textbf{t *} - Print a line for each started, ended or aborted trace (see also -jv).
  \item \textbf{b *} - Dump the traced bytecode.
  \item \textbf{i *} - Dump the IR (intermediate representation).
  \item \textbf{r} - Augment the IR with register/stack slots.
  \item \textbf{s} - Dump the snapshot map.
  \item \textbf{m *} - Dump the generated machine code.
  \item \textbf{x} - Print each taken trace exit.
  \item \textbf{X} - Print each taken trace exit and the contents of all registers.
  \item \textbf{a} - Print the IR of aborted traces, too.
\end{itemize}
The output format can be set with the following characters:
\begin{itemize}
   \item \textbf{T} - Plain text output.
   \item \textbf{A} - ANSI-colored text output
   \item \textbf{H} - Colorized HTML + CSS output.
\end{itemize}
The default output format is plain text. It's set to ANSI-colored text
if the COLORTERM variable is set.