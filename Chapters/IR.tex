%!TEX root = ../FYP_Dissertation.tex

\section{Snapshots}
\label{Sec:ir-snap}

Snapshot is an important mechanism used in trace compilation. The VM should
always be in a consistent state, meaning that all updates should respect the
original language semantics. However, to perform some trace optimization
(e.g. sinking optimization) this consistency is not respected. Instead
modification that should have occurred during a trace is recorded inside the
snapshots and those modification are replayed at trace exit. The Snapshot
mechanism is implemented in the snap.[hc] files, and you can find the
\emph{SnapShot} data-structure in lj\_jit.h. For details about
Snapshot usages and implementation refer to section of the wiki on sinking
optimization \cite{luajit-sink} and a mail on the subject \cite{luajit-mail-1}.
