%!TEX root = ../FYP_Dissertation.tex

The FFI (Foreign Function Interface) is one of LuaJIT extensions modules. It
allows to call external C functions and use C data structures from pure Lua code.
A very small (not representative) example can be seen bellow.

\begin{lstlisting}[style=LuaStyle]
local ffi = require("ffi")
ffi.cdef[[
  typedef struct points { int x,y,z; } points;
]]
local p1 = ffi.new("points", { 1, 2, 3 })
print(p1.x) --  1
p1.x = 11
print(p1.x) -- 11
\end{lstlisting}
There is some official documentation for FFI user on \emph{luajit.org}
where you can find the motivation for the FFI module \cite{ffi-motivation},
a small tutorial \cite{ffi-tuto}, the API documentation \cite{ffi-api} and the
FFI semantics \cite{ffi-semantics}. There is also an unofficial reflection
library for FFI ctypes \cite{ffi-reflect} and its documentation
\cite{ffi-reflect-doc} for anyone interested in exploring the ctype of a given cdata.

This chapter will present the internal implementation of the FFI and not its use.
The organisation of the information will follow the actual implementation files.

\paratitle{lib\_ffi.c:}\\
This file is the top level file of the FFI library. It contains the
implementation of the FFI API, the function that makes the connection between
lua and c using the standard Lua \emph{C API}.
It is also responsible for loading the FFI module (\emph{luaopen\_ffi}).
This file mainly use and connect together functionalities implemented in other
files. It is for example responsible to allocate and initialize the main state
(\emph{CTState}) explained bellow.

\paratitle{lj\_obj.h:}\\
From this file, we are only interested for this chapter by the \emph{GCcdata}
structure that is the garbage collectible object representing any c data use
through and with the FFI. You can see its composition bellow. The main thing to
see there is the \emph{ctypeid} which is the index of the ctype describing the
attached data (the payload follows the structure in memory).
\begin{lstlisting}[style=CStyle]
typedef struct GCcdata {
  GCHeader;
  uint16_t ctypeid;	/* C type ID. */
} GCcdata;
\end{lstlisting}

\paratitle{lj\_ctype.h:}\\
Bellow is the \emph{CType} data structure responsible for describing to the ffi
what kind of data the \emph{cdata} represent (e.g. variable, struct,
function etc...). A detailed breakdown of how the information are laid out can
be seen in Table \ref{tab:ffi-ctype}. The meaning of the abbreviations used in
the previouse table can be seen in Table \ref{tab:ffi-ctype2}.
\begin{lstlisting}[style=CStyle]
typedef struct CType {
  CTInfo info;   /* Type info. */
  CTSize size;   /* Type size or other info. */
  CTypeID1 sib;  /* Sibling element. */
  CTypeID1 next; /* Next element in hash chain. */
  GCRef name;    /* Element name (GCstr). */
} CType;
\end{lstlisting}

%          +----------------------------+--------+-------+-------+-------+
%          |            info            |        |       |       |       |
%          +----------------------------+  size  |  sid  | next  | name  |
%          |type|  flags | A  |   cid   |        |       |       |       |
% +-------------------------------------+--------------------------------+
% |size    | 4  |    8   | 4  |   16    |   32   |  16   |  16   | GCRef |
% |--------+----------------------------+--------+-------+-------+-------+
% |NUM     |0000|BFcvUL..| A  |    x    | size   |       | type  |       |
% |STRUCT  |0001|..cvu..V| A  |    x    | size   | field | name? | name? |
% |PTR     |0010|..cvR...| A  |   cid   | size   |       | type  |       |
% |ARRAY   |0011|VCcv...V| A  |   cid   | size   |       | type  |       |
% |VOID    |0100|..cv....| A  |    x    | size   |       | type  |       |
% |ENUM    |0101|........| A  |   cid   | size   | const | name? | name? |
% |FUNC    |0110|....VS..|..cc|   cid   | nargs  | field | name? | name? |
% |TYPEDEF |0111|........|....|   cid   |   x    |       | name  | name  |
% |ATTRIB  |1000|....|attrnum |   cid   | attr   | sib?  | type? |       |
% |FIELD   |1001|........|....|   cid   | offset | field |       | name? |
% |BITFIELD|1010|B.cvU| csz   |.bsz|.pos| offset | field |       | name? |
% |CONSTVAL|1011|..c.....|....|   cid   | value  | const | name  | name  |
% |EXTERN  |1100|........|....|   cid   |   x    | sib?  | name  | name  |
% |KW      |1101|........|....|   tok   | size   |       | name  | name  |
% +-------------------------------------+--------+-------+-------+-------+
\begin{table}[p]
\footnotesize
\centering
\caption{CType information summary}
\label{tab:ffi-ctype}
\begin{tabular}{l|c|c|c|c|c|l|c|c|c|c|}
\cline{2-11}
                               & \multicolumn{6}{c|}{info}                                                         & \multirow{2}{*}{size} & \multirow{2}{*}{sid} & \multirow{2}{*}{next} & \multirow{2}{*}{name} \\ \cline{2-7}
                               & type & \multicolumn{2}{c|}{flags}          & A         & \multicolumn{2}{c|}{cid} &                       &                      &                       &                       \\ \hline
\multicolumn{1}{|l|}{size}     & 4    & \multicolumn{2}{c|}{8}              & 4         & \multicolumn{2}{c|}{16}  & 32                    & 16                   & 16                    & GCRef                 \\ \hline
\multicolumn{1}{|l|}{NUM}      & 0000 & \multicolumn{2}{c|}{BFcvUL..}       & A         & \multicolumn{2}{c|}{}    & size                  &                      & type                  &                       \\
\multicolumn{1}{|l|}{STRUCT}   & 0001 & \multicolumn{2}{c|}{..cvu..V}       & A         & \multicolumn{2}{c|}{}    & size                  & field                & name                  & name                  \\
\multicolumn{1}{|l|}{PTR}      & 0010 & \multicolumn{2}{c|}{..cvR...}       & A         & \multicolumn{2}{c|}{cid} & size                  &                      & type                  &                       \\
\multicolumn{1}{|l|}{ARRAY}    & 0011 & \multicolumn{2}{c|}{$V^2$Ccv...V}   & A         & \multicolumn{2}{c|}{cid} & size                  &                      & type                  &                       \\
\multicolumn{1}{|l|}{VOID}     & 0100 & \multicolumn{2}{c|}{..cv....}       & A         & \multicolumn{2}{c|}{}    & size                  &                      & type                  &                       \\
\multicolumn{1}{|l|}{ENUM}     & 0101 & \multicolumn{2}{c|}{........}       & A         & \multicolumn{2}{c|}{cid} & size                  & const                & name                  & name                  \\
\multicolumn{1}{|l|}{FUNC}     & 0110 & \multicolumn{2}{c|}{....$V^3$S..}   & ..cc      & \multicolumn{2}{c|}{cid} & nargs                 & field                & name                  & name                  \\
\multicolumn{1}{|l|}{TYPEDEF}  & 0111 & \multicolumn{2}{c|}{........}       & ....      & \multicolumn{2}{c|}{cid} &                       &                      & name                  & name                  \\
\multicolumn{1}{|l|}{ATTRIB}   & 1000 & ....             & \multicolumn{2}{c|}{attrnum} & \multicolumn{2}{c|}{cid} & attr                  & sib                  & type                  &                       \\
\multicolumn{1}{|l|}{FIELD}    & 1001 & \multicolumn{2}{c|}{........}       & ....      & \multicolumn{2}{c|}{cid} & offset                & field                &                       & name                  \\
\multicolumn{1}{|l|}{BITFIELD} & 1010 & B.cvU            & \multicolumn{2}{c|}{csz}     & .bsz        & .pos       & offset                & field                &                       & name                  \\
\multicolumn{1}{|l|}{CONSTVAL} & 1011 & \multicolumn{2}{c|}{..c.....}       & ....      & \multicolumn{2}{c|}{cid} & value                 & const                & name                  & name                  \\
\multicolumn{1}{|l|}{EXTERN}   & 1100 & \multicolumn{2}{c|}{........}       & ....      & \multicolumn{2}{c|}{cid} &                       & sib                  & name                  & name                  \\
\multicolumn{1}{|l|}{KW}       & 1101 & \multicolumn{2}{c|}{........}       & ....      & \multicolumn{2}{c|}{tok} & size                  &                      & name                  & name                  \\ \hline
\end{tabular}
\end{table}

\begin{table}[p]
\footnotesize
\centering
\caption{My caption}
\label{tab:ffi-ctype2}
\begin{tabular}{ll|l|l|}
\hline
\multicolumn{2}{|c|}{\textit{\textbf{flags}}} & \multicolumn{2}{c|}{\textit{\textbf{A}}}                                    \\ \hline
\multicolumn{1}{|l|}{B}      & Boolean        & A                      & allignement of $2^A$ bytes                         \\
\multicolumn{1}{|l|}{F}      & Float          & cc                     & calling convention                                 \\
\multicolumn{1}{|l|}{c}      & const          & attrnum                & Attributes number (see CTA\_* in lj\_ctype.h)      \\
\multicolumn{1}{|l|}{v}      & volatile       & csz                    & size of the memory slot                            \\ \cline{3-4}
\multicolumn{1}{|l|}{U}      & Unsigned       & \multicolumn{2}{c|}{\textit{\textbf{cid}}}                                  \\ \cline{3-4}
\multicolumn{1}{|l|}{L}      & Long           & cid                    & child id                                           \\
\multicolumn{1}{|l|}{u}      & union          & bsz                    & number of bits of the bit-field                    \\
\multicolumn{1}{|l|}{V}      & VLA            & pos                    & starting position in bit inside the memory slot    \\
\multicolumn{1}{|l|}{R}      & reference      & tok                    & token id                                           \\ \cline{3-4}
\multicolumn{1}{|l|}{$V^2$}  & Vector         & \multicolumn{2}{c|}{\textit{\textbf{size}}}                                 \\ \cline{3-4}
\multicolumn{1}{|l|}{C}      & Complex        & size                   & size in bytes                                      \\
\multicolumn{1}{|l|}{$V^3$}  & Vararg         & nargs                  & number of fixed arguments of a function            \\
\multicolumn{1}{|l|}{S}      & SSE arguments  & offset                 & offset in bytes from the start of the struct       \\ \cline{1-2}
                             &                & value                  & the actual constant value                          \\
                             &                & attr                   & value of the attribute                             \\ \cline{3-4}
                             &                & \multicolumn{2}{c|}{\textit{\textbf{sib}}}                                  \\ \cline{3-4}
                             &                & field                  & function args or struc/union field                 \\
                             &                & const                  & constant value                                     \\
                             &                & sib                    & chain of attribute or object of the attribute      \\ \cline{3-4}
                             &                & \textit{\textbf{name}} & GCRef to a string containing the identifier.       \\ \cline{3-4}
                             &                & \textit{\textbf{next}} & chain for hash collisions in cts-\textgreater hash \\ \cline{3-4}
\end{tabular}
\end{table}

The most important struct of the FFI is the \emph{CTState} bellow. It contains
all the internalize ctype in the \emph{tab} table. \emph{finalizer} is a weak
keyed lua table (values can be garbage collected if the key is not referenced
elsewhere ) containing all finalizer registered with the \emph{ffi.gc} method.
\emph{miscmap} is a lua table mapping all metatable of ctypes registered using
the \emph{ffi.metatype} method in the negative CTypeID range and all callback
functions in the positive callback slot range. Any metatable added to miscmap is
definitive and never collected. \emph{hash} is an array used as an hash table
for quick CTypeID checks. It map both, the hashed name of named element and the
hashed type (info and size) for unnamed element to the corresponding CTypeID.
Collisions are handled in a linked list using the \emph{next} field of the
\emph{CType} struct.

\begin{adjustbox}{max width=\textwidth}
\begin{minipage}{\linewidth}
\begin{lstlisting}[style=CStyle]
typedef struct CTState {
  CType *tab;        /* C type table. */
  CTypeID top;       /* Current top of C type table. */
  MSize sizetab;     /* Size of C type table. */
  lua_State *L;      /* Lua state (for errors and allocations). */
  global_State *g;   /* Global state. */
  GCtab *finalizer;  /* Map of cdata to finalizer. */
  GCtab *miscmap;    /* Map -CTypeID->metatable and cb slot->func. */
  CCallback cb;      /* Temporary callback state. */
  CTypeID1 hash[...];/* Hash anchors for C type table. */
} CTState;
\end{lstlisting}
\end{minipage}
\end{adjustbox}

\paratitle{lj\_ctype.c:}\\
This file provide functions to manage \emph{CType}. It is divided into three
parts. The first one is the allocation, creation and internalization of
\emph{CType}. The second one is providing getters to get C type information.
The last one is  type representation, providing the necessary functions to
convert a \emph{CType} to an human readable string representation.
\emph{lj\_ctype\_repr} is the entry function that returns the internalize string
representation. It uses the struct \emph{CTRepr} bellow to create the
representation by appending/prepending characters through the pb/pe pointers
into the buffer. The main function is the \emph{ctype\_repr} that contains a
switch on the \emph{CType} info.

\begin{lstlisting}[style=CStyle]
typedef struct CTRepr {
  char *pb, *pe; /* Points to begining/end inside the buffer*/
  CTState *cts;  /* C type state. */
  lua_State *L;
  int needsp;    /* Next append needs an extra space character */
  int ok;        /* Indicate if buf is currently a valid type */
  char buf[...]; /* String buffer of the ctype being constructed */
} CTRepr;
\end{lstlisting}

\paratitle{lj\_cparse.h:}\\
The cparser is responsible for parsing the string making the C declarations
for types or external symbols. Its code structure is quite close to the
lua lexer/parser. Its principal struct is the \emph{CPState} bellow which is
similar to \emph{LexState}. In this struct, \emph{tmask} is a mask constraining
the possible ctype of the next identifier. The \emph{mode} define the behavior
of the parser with respect to the imput. It's different behavior can be of the
type: accepting multiple declaration, skiping error, accept/reject abstract
declarators, accept/reject implicit declarators etc... (see CPARSE\_MODE\_* for
full definition).

\begin{adjustbox}{bgcolor=backgroundColour, max width=\textwidth}
\begin{lstlisting}[style=CStyle]
typedef struct CPState {
  CPChar c;               /* Current character. */
  CPToken tok;            /* Current token. */
  CPValue val;            /* Token value. */
  GCstr *str;             /* Interned string of identifier/keyword. */
  CType *ct;              /* C type table entry. */
  const char *p;          /* Current position in input buffer. */
  SBuf sb;                /* String buffer for tokens. */
  lua_State *L;           /* Lua state. */
  CTState *cts;           /* C type state. */
  TValue *param;          /* C type parameters. ($xyz)*/
  const char *srcname;    /* Current source name. */
  BCLine linenumber;      /* Input line counter. */
  int depth;              /* Recursive declaration depth. */
  uint32_t tmask;         /* Type mask for next identifier. */
  uint32_t mode;          /* C parser mode. */
  uint8_t packstack[...]; /* Stack for pack pragmas. */
  uint8_t curpack;        /* Current position in pack pragma stack. */qerf
} CPState;
\end{lstlisting}
\end{adjustbox}

\paratitle{lj\_cparse.c:}\\
This file contains the code of a simple lexer and a simplified, not valid c
parser. It use the \emph{CPState} describe abose for the parsing of the input
string and the \emph{CPDecl} structure bellow for the construction of the
corresponding CType. During parsing the typdef chain is unrolled (typdef are
still internalized for future reference but are not chained to the created
ctype.)

\begin{lstlisting}[style=CStyle]
typedef struct CPDecl {
  CPDeclIdx top;     /* Top of declaration stack. */
  CPDeclIdx pos;     /* Insertion position in declaration chain. */
  CPDeclIdx specpos; /* Saved position for declaration specifier. */
  uint32_t mode;     /* Declarator mode (same as CPState) */
  CPState *cp;       /* C parser state. */
  GCstr *name;       /* Name of declared identifier (if direct). */
  GCstr *redir;      /* Redirected symbol name. */
  CTypeID nameid;    /* Existing typedef for declared identifier. */
  CTInfo attr;       /* Attributes. */
  CTInfo fattr;      /* Function attributes. */
  CTInfo specattr;   /* Saved attributes. */
  CTInfo specfattr;  /* Saved function attributes. */
  CTSize bits;       /* Field size in bits (see Ctype bsz). */
  CType stack[...];  /* Type declaration stack. */
} CPDecl;
\end{lstlisting}

\paratitle{lj\_cdata.c:}\\
This file contains the functions responsible for doing cdata management, such as
allocations, free, finilizer, getter, setter and indexing.

\paratitle{lj\_cconv.c:}\\
This file is responsible for ctype conversion. It is divided in 5 parts: C type
compatibility checks, C type to C type conversion, C type to TValue conversion,
(from c to lua : i.e returned values), TValue to C type conversion (from lua to
c: i.e passed arguments), and initializing C type with TValues (Initialization
of struct/union/array with lua object)

\paratitle{lj\_carith.c:}\\%
This file contains the implementation far all built-in cdata arithmetics, such
as pointers arithmetics and integer arithmetics. It mainly manipulate some
\emph{CDArith} structure shown bellow.
\begin{lstlisting}[style=CStyle]
typedef struct CDArith {
  uint8_t *p[2]; /* data  of the two operands */
  CType  *ct[2]; /* ctype of the two operands */
} CDArith;
\end{lstlisting}

\paratitle{lj\_ccall.c:}\\
This file contains the code handling calls to c function. It does structure/array
register classification (see CCALL\_RCL\_*), computing how it can be passed
as argument/return values (in gp register, sse register or memory). it handle
then the decomposition/packing depending of the calling convention picked.

\paratitle{lj\_ccall.h:}\\
Bellow is the main structure use for c function call.
\begin{lstlisting}[style=CStyle]
typedef struct CCallState {
  void (*func)(void); /* Pointer to called function. */
  uint32_t spadj;     /* Stack pointer adjustment. */
  uint8_t nsp;        /* Number of stack slots. */
  uint8_t retref;     /* Return value by reference. */
  uint8_t ngpr;       /* Number of arguments in GPRs. */
  uint8_t nfpr;       /* Number of arguments in FPRs. */
  [...]
  FPRArg fpr[...];    /* Arguments/results in FPRs. (SSE) */
  GPRArg gpr[...];    /* Arguments/results in GPRs. */
  GPRArg stack[...];  /* Stack slots. */
} CCallState;
\end{lstlisting}

\paratitle{lj\_ccallback.c:}\\
This file contains the FFI C callback handling. The principal structure is the
\emph{CCallback} bellow (see \emph{lj\_ctype.h}) mainly use through the cb field of the \emph{CTState}
structure. Each callback is associated with a unique cb slot and the
\emph{cts-$>$miscmap} contains the mapping between cb slot and function
pointers. The \emph{cts.cb.cbid} is a table mapping cb slot to the corresponding
CTypeID. \emph{cts.cb.mcode} is an mmap'ed executable page that contains a push
of slot id and a jump to \emph{lj\_vm\_ffi\_callback} (the entry in this page
correspond to the cb address provided to the c code). Bellow is a description
of the call order when a callback is called.

\begin{itemize}
	\item c call a callback address
	\item arrive in the callback mcode page (push the appropriate slot id and call lj\_vm\_ffi\_callback)
	\item arrive in lj\_vm\_ffi\_callback that charge in registers the callback.fpr/gpr + stack in registers
	\item arrive in lj\_ccallback\_enter that prepare the lua state and do conversion of argument from c to lua type.
	\item arrive in lj\_vm\_ffi\_callback : execute the callback with this lua state
	\item lua callback
	\item arrive in $|->$cont\_ffi\_callback: call lj\_ccallback\_leave
	\item arrive in lj\_ccallback\_leave convert return value from lua to c.
	\item arrive in $|->$cont\_ffi\_callback: return to c code with the return value in register/stack.
	\item arrive in c code.
\end{itemize}

\begin{lstlisting}[style=CStyle]
typedef struct CCallback {
  FPRCBArg fpr[...]; /* Arguments/results in FPRs. */
  intptr_t gpr[...]; /* Arguments/results in GPRs. */
  intptr_t *stack;   /* Pointer to arguments on stack. */
  void *mcode;       /* Machine code for callback func. pointers. */
  CTypeID1 *cbid;    /* Callback type table. */
  MSize sizeid;      /* Size of callback type table. */
  MSize topid;       /* Highest unused callback type table slot. */
  MSize slot;        /* Current callback slot. */
} CCallback;
\end{lstlisting}

\paratitle{lj\_clib.c:}\\
This file contain the necessary code to load/unload ffi library. It also handle
the indexing of the external library using named symbol. It use platform
specific tool to explore the exposed symbol. Every symbol is resolved only once
and cached in the \emph{CLibrary} cache table.

\paratitle{lj\_clib.h:}\\
\begin{lstlisting}[style=CStyle]
typedef struct CLibrary {
  void *handle; /* Opaque handle for dynamic library loader. */
  GCtab *cache; /* Cache for resolved symbols. Anchored in ud->env. */
} CLibrary;
\end{lstlisting}
