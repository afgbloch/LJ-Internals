%!TEX root = ../FYP_Dissertation.tex

The FFI (Foreign Function Interface) is one of LuaJIT extensions modules. It
allows to call external C functions and use C data structures from pure Lua code.
A very small (not representative) example can be seen bellow.

\begin{lstlisting}[style=myLuastyle]
local ffi = require("ffi")
ffi.cdef[[
  typedef struct points { int x,y,z; } points;
]]
local p1 = ffi.new("points", { 1, 2, 3 })
print(p1.x) --  1
p1.x = 11
print(p1.x) -- 11
\end{lstlisting}
There is some official documentation for FFI user on \emph{luajit.org}
where you can find the motivation for the FFI module \cite{ffi-motivation},
a small tutorial \cite{ffi-tuto}, the API documentation \cite{ffi-api} and the
FFI semantics \cite{ffi-semantics}. There is also an unofficial reflection
library for FFI ctypes \cite{ffi-reflect} and its documentation
\cite{ffi-reflect-doc} for anyone interested in exploring the ctype of a given cdata.

This chapter will present the internal implementation of the FFI and not its use.
The organisation of the information will follow the actual implementation files.

\paratitle{lib\_ffi.c:}\\
This file is the top level file of the FFI library. It contains the
implementation of the FFI API, the function that makes the connection between
lua and c using the standard Lua \emph{C API}.
It is also responsible for loading the FFI module (\emph{luaopen\_ffi}).
This file mainly use and connect together functionalities implemented in other
files. It is for example responsible to allocate and initialize the main state
(\emph{CTState}) explained bellow.

\paratitle{lj\_obj.h:}\\
From this file, we are only interested for this chapter by the \emph{GCcdata}
structure that is the garbage collectible object representing any c data use
through and with the FFI. You can see its composition bellow. The main thing to
see there is the \emph{ctypeid} which is the index of the ctype describing the
attached data (the payload follows the structure in memory).
\begin{lstlisting}[style=CStyle]
typedef struct GCcdata {
  GCHeader;
  uint16_t ctypeid;	/* C type ID. */
} GCcdata;
\end{lstlisting}

\paratitle{lj\_ctype.h:}\\
Bellow is the \emph{CType} data structure responsible for describing to the ffi
what kind of data the \emph{cdata} represent (e.g. variable, struct,
function etc...). A detailed breakdown of how the information are laid out can
be seen in Table \ref{tab:ffi-ctype}. The meaning of the abbreviations used in
the previouse table can be seen in Table \ref{tab:ffi-ctype2}.
\begin{lstlisting}[style=CStyle]
typedef struct CType {
  CTInfo info;   /* Type info. */
  CTSize size;   /* Type size or other info. */
  CTypeID1 sib;  /* Sibling element. */
  CTypeID1 next; /* Next element in hash chain. */
  GCRef name;    /* Element name (GCstr). */
} CType;
\end{lstlisting}

%          +----------------------------+--------+-------+-------+-------+
%          |            info            |        |       |       |       |
%          +----------------------------+  size  |  sid  | next  | name  |
%          |type|  flags | A  |   cid   |        |       |       |       |
% +-------------------------------------+--------------------------------+
% |size    | 4  |    8   | 4  |   16    |   32   |  16   |  16   | GCRef |
% |--------+----------------------------+--------+-------+-------+-------+
% |NUM     |0000|BFcvUL..| A  |    x    | size   |       | type  |       |
% |STRUCT  |0001|..cvu..V| A  |    x    | size   | field | name? | name? |
% |PTR     |0010|..cvR...| A  |   cid   | size   |       | type  |       |
% |ARRAY   |0011|VCcv...V| A  |   cid   | size   |       | type  |       |
% |VOID    |0100|..cv....| A  |    x    | size   |       | type  |       |
% |ENUM    |0101|........| A  |   cid   | size   | const | name? | name? |
% |FUNC    |0110|....VS..|..cc|   cid   | nargs  | field | name? | name? |
% |TYPEDEF |0111|........|....|   cid   |   x    |       | name  | name  |
% |ATTRIB  |1000|....|attrnum |   cid   | attr   | sib?  | type? |       |
% |FIELD   |1001|........|....|   cid   | offset | field |       | name? |
% |BITFIELD|1010|B.cvU| csz   |.bsz|.pos| offset | field |       | name? |
% |CONSTVAL|1011|..c.....|....|   cid   | value  | const | name  | name  |
% |EXTERN  |1100|........|....|   cid   |   x    | sib?  | name  | name  |
% |KW      |1101|........|....|   tok   | size   |       | name  | name  |
% +-------------------------------------+--------+-------+-------+-------+
\begin{table}[H]
\footnotesize
\centering
\caption{CType information summary}
\label{tab:ffi-ctype}
\begin{tabular}{l|c|c|c|c|c|l|c|c|c|c|}
\cline{2-11}
                               & \multicolumn{6}{c|}{info}                                                         & \multirow{2}{*}{size} & \multirow{2}{*}{sid} & \multirow{2}{*}{next} & \multirow{2}{*}{name} \\ \cline{2-7}
                               & type & \multicolumn{2}{c|}{flags}          & A         & \multicolumn{2}{c|}{cid} &                       &                      &                       &                       \\ \hline
\multicolumn{1}{|l|}{size}     & 4    & \multicolumn{2}{c|}{8}              & 4         & \multicolumn{2}{c|}{16}  & 32                    & 16                   & 16                    & GCRef                 \\ \hline
\multicolumn{1}{|l|}{NUM}      & 0000 & \multicolumn{2}{c|}{BFcvUL..}       & A         & \multicolumn{2}{c|}{}    & size                  &                      & type                  &                       \\
\multicolumn{1}{|l|}{STRUCT}   & 0001 & \multicolumn{2}{c|}{..cvu..V}       & A         & \multicolumn{2}{c|}{}    & size                  & field                & name                  & name                  \\
\multicolumn{1}{|l|}{PTR}      & 0010 & \multicolumn{2}{c|}{..cvR...}       & A         & \multicolumn{2}{c|}{cid} & size                  &                      & type                  &                       \\
\multicolumn{1}{|l|}{ARRAY}    & 0011 & \multicolumn{2}{c|}{$V^2$Ccv...V}   & A         & \multicolumn{2}{c|}{cid} & size                  &                      & type                  &                       \\
\multicolumn{1}{|l|}{VOID}     & 0100 & \multicolumn{2}{c|}{..cv....}       & A         & \multicolumn{2}{c|}{}    & size                  &                      & type                  &                       \\
\multicolumn{1}{|l|}{ENUM}     & 0101 & \multicolumn{2}{c|}{........}       & A         & \multicolumn{2}{c|}{cid} & size                  & const                & name                  & name                  \\
\multicolumn{1}{|l|}{FUNC}     & 0110 & \multicolumn{2}{c|}{....$V^3$S..}   & ..cc      & \multicolumn{2}{c|}{cid} & nargs                 & field                & name                  & name                  \\
\multicolumn{1}{|l|}{TYPEDEF}  & 0111 & \multicolumn{2}{c|}{........}       & ....      & \multicolumn{2}{c|}{cid} &                       &                      & name                  & name                  \\
\multicolumn{1}{|l|}{ATTRIB}   & 1000 & ....             & \multicolumn{2}{c|}{attrnum} & \multicolumn{2}{c|}{cid} & attr                  & sib                  & type                  &                       \\
\multicolumn{1}{|l|}{FIELD}    & 1001 & \multicolumn{2}{c|}{........}       & ....      & \multicolumn{2}{c|}{cid} & offset                & field                &                       & name                  \\
\multicolumn{1}{|l|}{BITFIELD} & 1010 & B.cvU            & \multicolumn{2}{c|}{csz}     & .bsz        & .pos       & offset                & field                &                       & name                  \\
\multicolumn{1}{|l|}{CONSTVAL} & 1011 & \multicolumn{2}{c|}{..c.....}       & ....      & \multicolumn{2}{c|}{cid} & value                 & const                & name                  & name                  \\
\multicolumn{1}{|l|}{EXTERN}   & 1100 & \multicolumn{2}{c|}{........}       & ....      & \multicolumn{2}{c|}{cid} &                       & sib                  & name                  & name                  \\
\multicolumn{1}{|l|}{KW}       & 1101 & \multicolumn{2}{c|}{........}       & ....      & \multicolumn{2}{c|}{tok} & size                  &                      & name                  & name                  \\ \hline
\end{tabular}
\end{table}

\begin{table}[H]
\footnotesize
\centering
\caption{My caption}
\label{tab:ffi-ctype2}
\begin{tabular}{ll|l|l|}
\hline
\multicolumn{2}{|c|}{\textit{\textbf{flags}}} & \multicolumn{2}{c|}{\textit{\textbf{A}}}                                    \\ \hline
\multicolumn{1}{|l|}{B}      & Boolean        & A                      & allignement of $2^A$ bytes                         \\
\multicolumn{1}{|l|}{F}      & Float          & cc                     & calling convention                                 \\
\multicolumn{1}{|l|}{c}      & const          & attrnum                & Attributes number (see CTA\_* in lj\_ctype.h)      \\
\multicolumn{1}{|l|}{v}      & volatile       & csz                    & size of the memory slot                            \\ \cline{3-4}
\multicolumn{1}{|l|}{U}      & Unsigned       & \multicolumn{2}{c|}{\textit{\textbf{cid}}}                                  \\ \cline{3-4}
\multicolumn{1}{|l|}{L}      & Long           & cid                    & child id                                           \\
\multicolumn{1}{|l|}{u}      & union          & bsz                    & number of bits of the bit-field                    \\
\multicolumn{1}{|l|}{V}      & VLA            & pos                    & starting position in bit inside the memory slot    \\
\multicolumn{1}{|l|}{R}      & reference      & tok                    & token id                                           \\ \cline{3-4}
\multicolumn{1}{|l|}{$V^2$}  & Vector         & \multicolumn{2}{c|}{\textit{\textbf{size}}}                                 \\ \cline{3-4}
\multicolumn{1}{|l|}{C}      & Complex        & size                   & size in bytes                                      \\
\multicolumn{1}{|l|}{$V^3$}  & Vararg         & nargs                  & number of fixed arguments of a function            \\
\multicolumn{1}{|l|}{S}      & SSE arguments  & offset                 & offset in bytes from the start of the struct       \\ \cline{1-2}
                             &                & value                  & the actual constant value                          \\
                             &                & attr                   & value of the attribute                             \\ \cline{3-4}
                             &                & \multicolumn{2}{c|}{\textit{\textbf{sib}}}                                  \\ \cline{3-4}
                             &                & field                  & function args or struc/union field                 \\
                             &                & const                  & constant value                                     \\
                             &                & sib                    & chain of attribute or object of the attribute      \\ \cline{3-4}
                             &                & \textit{\textbf{name}} & GCRef to a string containing the identifier.       \\ \cline{3-4}
                             &                & \textit{\textbf{next}} & chain for hash collisions in cts-\textgreater hash \\ \cline{3-4}
\end{tabular}
\end{table}

The most important struct of the FFI is the \emph{CTState} bellow. It contains
all the internalize ctype in the \emph{tab} table. \emph{finalizer} is a weak
keyed lua table (values can be garbage collected if the key is not referenced
elsewhere ) containing all finalizer registered with the \emph{ffi.gc} method.
\emph{miscmap} is a lua table mapping all metatable of ctypes registered using
the \emph{ffi.metatype} method in the negative CTypeID range and all callback
functions in the positive callback slot range. Any metatable added to miscmap is
definitive and never collected. \emph{hash} is an array used as an hash table
for quick CTypeID checks. It map both, the hashed name of named element and the
hashed type (info and size) for unnamed element to the corresponding CTypeID.
Collisions are handled in a linked list using the \emph{next} field of the
\emph{CType} struct.

\begin{lstlisting}[style=CStyle]
typedef struct CTState {
  CType *tab;        /* C type table. */
  CTypeID top;       /* Current top of C type table. */
  MSize sizetab;     /* Size of C type table. */
  lua_State *L;      /* Lua state (for errors and allocations). */
  global_State *g;   /* Global state. */
  GCtab *finalizer;  /* Map of cdata to finalizer. */
  GCtab *miscmap;    /* Map -CTypeID->metatable and cb slot->func. */
  CCallback cb;      /* Temporary callback state. */
  CTypeID1 hash[...];/* Hash anchors for C type table. */
} CTState;
\end{lstlisting}