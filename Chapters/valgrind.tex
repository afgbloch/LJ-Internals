%!TEX root = ../FYP_Dissertation.tex

Valgrind is an instrumentation framework for building dynamic analysis tools.
There are Valgrind tools that can automatically detect many memory management
and threading bugs, and profile programs in detail (cf \cite{valgrind}). To be
able to debug your c code used through ffi with Valgrind, LuaJIT needs to be
compiled with specific options. Some information on that, can be found in the
LuaJIT \emph{Makefile} but a summary of the way to proceed for linux is describe
here. First we need to uncomment in \emph{src/makefile}, \emph{CCDEBUG} to
allow for debug information, \emph{DLUAJIT\_USE\_SYSMALLOC} to use the system
allocator (to get useful memcheck information), \emph{DLUAJIT\_USE\_VALGRIND}
and \emph{DLUAJIT\_ENABLE\_GC64} if run on a x64 platform. After recompiling
with those options, you will need the suppression file \emph{src/lj.supp} that
contains known false positive for memcheck to run a valgrind command such as the
following.

\begin{lstlisting}[style=CStyle]
valgrind
		--tool=memcheck
		--leak-check=full
		--track-origins=yes
		--log-file='log.txt'
		--suppressions=lj.supp
		./luajit myprogram
\end{lstlisting}


