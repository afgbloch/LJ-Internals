%!TEX root = ../FYP_Dissertation.tex

We can differentiate in the code, three types of optimizations.
First of all there are the optimizations present in the optimization engine.
They are implemented in the \emph{lj\_opt\_*.c} files and thus easy to spot out.
Those optimization are of two kind, global optimization that are run on the
entire IR at once at the end of the recording phase (See \ref{Sec:opt-dce}, \ref{Sec:opt-loop}, \ref{Sec:opt-split}, \ref{Sec:opt-sinking}) and the local
optimizations that are applied while recording a trace (See \ref{Sec:narrowing},
\ref{Sec:mao}, \ref{Sec:fold}).
Finally, there is a plethora of optimization and heuristic applied here and
there (See ).\todo{add references}
% ------------------------------------------------------------------------------
% Optimization Engine : Global Optimizations (end of recording phase)
% ------------------------------------------------------------------------------

\section{Dead code elimination}
\label{Sec:opt-dce}

\section{Loop optimizations}
\label{Sec:opt-loop}

\section{Split optimizations}
\label{Sec:opt-split}

\section{Sinking optimizations}
\label{Sec:opt-sinking}

% ------------------------------------------------------------------------------
% Optimization Engine : Local Optimizations (During recording phase)
% ------------------------------------------------------------------------------

\section{Narrowing optimizations}
\label{Sec:narrowing}

\section{Memory access optimizations}
\label{Sec:mao}

\section{Fold optimizations}
\label{Sec:fold}

% ------------------------------------------------------------------------------
% Embeded Optimizations
% ------------------------------------------------------------------------------