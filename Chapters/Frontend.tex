%!TEX root = ../FYP_Dissertation.tex

\section{Lexer}
\label{Sec:Lexer}

The lexer is implemented in the \emph{lj\_lex.c} and \emph{lj\_lex.h} files.
It use \emph{LexState} as a principal data structure. The user-provided
\emph{rfunc} function is used to read a chunk of data to process. It is accessed
through the \emph{p} and \emph{pe} pointers. The main function is
\emph{lex\_scan} that dispatches the work to other functions depending of the
type of data to be processed (comment, string literal, long string, numbers
etc...). TValues (\emph{tokval}, \emph{lookaheadval}) are used to store the
token values were \emph{LexToken} (\emph{tok}, \emph{lookahead}) determine the
type. The string buffer (\emph{sb}) is used to accumulate characters of a future
string before internalizing it. All lua keyword are internalized as a string at
the very beginning, GCstr has the field \emph{reserved} for marking them.

\begin{lstlisting}[style=CStyle]
typedef struct LexState {
  struct FuncState *fs; /* Current FuncState. Defined in lj_parse.c. */
  struct lua_State *L;  /* Lua state. */
  TValue tokval;        /* Current token value. */
  TValue lookaheadval;  /* Lookahead token value. */
  const char *p;        /* Current position in input buffer. */
  const char *pe;       /* End of input buffer. */
  LexChar c;            /* Current character. */
  LexToken tok;         /* Current token. */
  LexToken lookahead;   /* Lookahead token. */
  SBuf sb;              /* String buffer for tokens. */
  lua_Reader rfunc;     /* Reader callback. */
  void *rdata;          /* Reader callback data. */
  BCLine linenumber;    /* Input line counter. */
  BCLine lastline;      /* Line of last token. */
  GCstr *chunkname;     /* Current chunk name (interned string). */
  const char *chunkarg; /* Chunk name argument. */
  const char *mode;     /* Allow loading bytecode (b) and/or source text (t). */
  VarInfo *vstack;      /* Stack for names and extents of local variables. */
  MSize sizevstack;     /* Size of variable stack. */
  MSize vtop;           /* Top of variable stack. */
  BCInsLine *bcstack;   /* Stack for bytecode instructions/line numbers. */
  MSize sizebcstack;    /* Size of bytecode stack. */
  uint32_t level;       /* Syntactical nesting level. */
} LexState;
\end{lstlisting}

\section{Parser}
\label{Sec:Parser}

The parser of luaJIT is implemented in \emph{lj\_parse.c} and \emph{lj\_parse.h}
files. it also use \emph{LexState} as the principal data-structure. it is the on
responsible of calling the lexer to get tokens. LuaJIT doesn't really have an
abstract syntax tree representation of the parsed code. Instead it directly
generates the bytecode on the fly using helpers from \emph{lj\_bc.h}.

\section{Bytecode frontend}
\label{Sec:bc-frontend}

Another frontend feature provided by luaJIT is the possibility to save and load
bytecode directly, allowing to skip the lexer and parser phase.

The writing part is handled by the module \emph{bcsave.lua} that use the
\emph{lj\_bcwrite} function from \emph{lj\_bcwrite.c} to generate the data to be written.

The reading part is done by the code in \emph{lj\_bcread.c} file. When it is
detected that the input file is a bc dump instead of a plain lua code,
\emph{cpparser} from \emph{lj\_load.c} calls \emph{lj\_bcread} instead of
\emph{lj\_parse} normally. This reader also use \emph{LexState} as the principal
data-structure.

