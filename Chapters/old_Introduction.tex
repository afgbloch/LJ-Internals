This document is a brief introduction to \LaTeX. It is meant to be used as a quick start to creating the final year dissertation. It is not meant to be a comprehensive guide to the use of \LaTeX. For that, you could refer to~\cite{Oetiker2015}. A cheat sheet \footnote{\url{https://wch.github.io/latexsheet/latexsheet-a4.pdf}} is also available for quick reference.

\section{Using images in \LaTeX}

Images can be inserted using the \texttt{figure} environment as follows:

\begin{figure}[t]
	\centering
	\subfigure[image 1]{\includegraphics[width = 5cm]{./Pictures/InconsistentShadow} \label{subfig:First}} \quad
		\subfigure[image 2]{\includegraphics[width = 5cm]{./Pictures/InconsistentShadow}\subfigure{second}}
	\caption[A short caption]{The caption command allows you to define an optional caption, in the square brackets, which is used in the list of figures. This comes in handy when the figure requires a long caption as follows:\label{Fig:Image1}}
\end{figure}

Figure~\ref{subfig:First}

\noindent Figure~\ref{Fig:Image1} The \texttt{label} command allows you to assign a label to the figure so that you can reference it like so: Figure

The caption command allows you to define an optional caption, in the square brackets, which is used in the list of figures. This comes in handy when the figure requires a long caption as follows:

\begin{figure}[h!]
	\centering
	\includegraphics[width=8cm]{./Pictures/InconsistentShadow}
	\caption{An example of an image with a long caption which is shortened in the table of contents. This is a long caption to illustrate why you would need to specify the second caption in the square brackets in the caption command.}\label{Fig:ImageLongCaption}
\end{figure}
\subsection{Side by side images}

The \texttt{subfigure} package allows you to place figures side-by-side as follows:


\begin{figure}[h!]
  \centering
  \subfigure[Image 1]{\includegraphics[width=4cm]{./Pictures/InconsistentShadow}\label{SubFig:A}}\quad
  \subfigure[Image 2]{\includegraphics[width=4cm]{./Pictures/InconsistentShadow}\label{SubFig:B}}\\
  \subfigure[Image 3]{\includegraphics[width=4cm]{./Pictures/InconsistentShadow}\label{SubFig:C}}
  \caption[An example of two side-by-side images]{Placing images side-by-side.}\label{Fig:SideBySide}
\end{figure}



\noindent If you label the individual image, you could also reference it as an individual: Figure~\ref{SubFig:A} or as a whole Figure~\ref{Fig:SideBySide}.  You could place a caption with the individual image by placing text in the square brackets after the subfigure, or you could omit them completely and the figure will not be labelled.

\textbf{Note that usually, figures are placed at the top of the page and to do so, the placement instruction should be \texttt{t}. Figures are placed using \texttt{h!} only for illustrative purposes. }

\section{Using tables in \LaTeX}
\label{sec:label}
\subsection{a new subsection}
\subsubsection{a sub sub section}
Tables can also be inserted easily into your document. The \texttt{booktabs} package is recommended as it allows you to typeset nice-looking tables. Like figures, tables should be captioned, but unlike figures, the caption is usually placed at the top of the page. Tables can be labelled and referenced in the same way as figures. Table~\ref{Tab:TabEx1} provides one such example. This table has examples of how \LaTeX handles merged rows and columns.

\begin{table}
	\centering
	\caption{A table caption}
	\begin{tabular}{l  r} \toprule[2pt]
		
		pets & quantity\\ \midrule
		dog & 1\\
		cat & 1\\ \bottomrule[2pt]
	\end{tabular}	
	
\end{table}



\begin{table}[h!]
  \caption[An example of a table]{An example of a table which has examples of merged columns and merged rows.}\label{Tab:TabEx1}
       \resizebox{14cm}{!} {
   \centering
    \begin{tabular}{c c  c  c  c }
    \toprule[2pt]
    \textbf{Drawing} & \multicolumn{2}{c}{\textbf{Method 1}} & \multicolumn{2}{c}{\textbf{Method 2}} \\[0.1cm] \cline{2-5}
                     & Some result & Some other result & Some result & Some other result \\ \midrule
    4.1              & 100 & 0  & 51 & 49 \\
    4.2              & 100 & 0  & 57 & 43 \\
    4.3              & 100 & 0  & 39 & 61 \\
    \multirow{ 2}{*}{4.4} & 90 & 10 & 39 & 61\\
                                          & 80 & 20 & 39 & 61\\
   \bottomrule[2pt]
   \end{tabular}}
\end{table}

It is also possible to use tables which require more text in their columns. This can be done by using the \texttt{paragraph} column setting as illustrated in Table~\ref{Tab:TabEx2}.

\begin{table}[h!]
  \caption[Another example of a table]{An example of a table which has  a paragraph column.}\label{Tab:TabEx2}
   \centering
    \begin{tabular}{p{8cm} c    }
    \toprule[2pt]
    \emph{Some random text} & \textbf{Result} \\\midrule
    Papalla jolloin me et kelpasi ei hyllyen en. Oma ulos juon ensi toru toi ole kone han.  & 100  \\[0.8cm]
     Syotin ei tekisi puista ai tiedat olette olivat he. Herra minka ei en se ai masto.  & 80\\
   \bottomrule[2pt]
   \end{tabular}
\end{table}

\section{Using equations in \LaTeX}
\LaTeX is also very good in handling equations, be it small equations which can fit in text like this: $y = mx + c$ or more complex equations which require to be displayed on separate lines such as this:


\begin{equation}
f\left(x\right) = \Gamma \exp{\left(- { \frac{(x-b)^2 }{ 2 c^2} } \right)}
\end{equation}


\noindent It is possible to label equations so that you can refer to them in the same way we did for the figures and tables, that is,  Equation~\ref{Eq:2DGaussian}.

If your equation has two or more parts to it, then you could use the \texttt{eqnarray} environment, or better still, the \texttt{align} environment as follows:

\begin{align}
2x - 5y &=  8 \nonumber\\
3x + 9y + 4z &=  -12 \label{Eq:First}
\end{align}

\noindent If you do not want to have both equations numbered, then you could use the \texttt{nonumber} option Equation~\ref{Eq:First}

\begin{align}
\nonumber 2x - 5y &=  8 \\
3x + 9y &=  -12
\end{align}

Equations that contain arrays can also be easily included

\begin{equation}\label{Eq:array}
  f(x)  =  \left(\begin{array}{cc}
                                \sin \theta & \cos \theta \\
                                \cos \theta & \sin \theta \end{array}\right)
                                %
                                 \left(\begin{array}{c}
                                x \\
                                y \end{array}\right)
\end{equation}

\noindent and if your equations contains cases, these can be written as follows:

\begin{align}
\chi(m)  &= \begin{cases}
    1, & \text{if }   \kappa_j(m)  \neq 0\\
    0, & \text{otherwise}
\end{cases}
\end{align}

\section{Populating the acronyms and symbols pages}
In your document, you may also wish to make use of symbols and acronyms. \LaTeX allows you to define these as you're writing your document and takes care to collect these under the appropriate lists in the front matter. The two examples below are meant to give you an idea of how to use the \texttt{glossaries} package. For a more detailed description, refer to~\cite{Talbot2015}.


\newglossaryentry{emptyset}
{
name={\ensuremath{\emptyset}},
description={the empty set}
}

\subsection{The list of Symbols}
Symbols can be defined by giving the symbol a name, telling \LaTeX what it should look like, and giving it a definition as follows:

\begin{verbatim}
\newglossaryentry{emptyset}
{
name={\ensuremath{\emptyset}},
description={the empty set}
}
\gls{emptyset}
\end{verbatim}

 The symbol can then be referenced in text using \verb|\gls{emptyset}|. For example, we could say : An empty set, denoted by \gls{emptyset}, is a set which contains nothing.


\subsection{The list of Acronyms}
\newacronym{EEG}{EEG}{electroencephalogram}

 It is also possible to define acronyms using \\
 \begin{verbatim}
 \newacronym{EEG}{EEG}{electroencephalogram}
 \end{verbatim}

 Here the glossaries package will take care of making sure that the first time this is used, the acronym will be spelt out in full, while all other times you will get just the abbreviations. For example:  Electric charge from the scalp can be measured using an \gls{EEG}. The \glspl{EEG} does not hurt but makes you look funny.

 Depending on the nature of your acronyms, you may decide to go all fancy for example:

 \newacronym{npar}{NPAR}{\textbf{n}on-\textbf{p}hotorealistic \textbf{a}nimation \textbf{r}endering}

 \Gls{npar} is a cool research area. Through \gls{npar} you can create different renderings of your photos.

 \section{Creating Lists}
 \LaTeX also provides a very neat way of including lists in your documents. There are three types of lists that can be used as detailed hereunder. Note that all lists can have nested lists within each item if this is necessary.

 \subsection{Bullet lists}
 These are created using the \texttt{itemize} environment:

 \begin{itemize}
   \item Item 1
   \item Item 2
   \item Item 3
   \begin{itemize}
   	\item Item 3a
   \end{itemize}
 \end{itemize}

 \subsection{Numbered lists}
 These are created using the \texttt{enumerate} environment:
 \begin{enumerate}
   \item Numbered item 1
   \item Numbered item 2
   \item Numbered item 3
   \begin{enumerate}
   	\item 1a
   \end{enumerate}
 \end{enumerate}

 \subsection{Descriptive list}
 This is a list created using the \texttt{description} environment and comes handy when you have a list of items to describe:

 \begin{description}
   \item[Domestic cat:]  a small, usually furry, domesticated, and carnivorous mammal. It usually has four legs, two ears and a tail.
   \item[Tiger:] a slightly bigger and more aggressive version of the domestic cat. Keeping one at home can be considered crazy.
 \end{description}

 \section{Definitions, Lemmas and Proofs}

Sometimes, it may be necessary to give mathematical theory and this can be achieved through the use of Lemmas and Definitions as follows:

 \begin{definition}
Let $\| a'' \| \ni 0$ be arbitrary.  We say a function ${l_{\Xi,Y}}$ is \textbf{Poncelet--Kummer} if it is co-measurable.
\end{definition}

\begin{lemma}
Let $\hat{m} \ge {\Phi_{r,\omega}}$ be arbitrary.  Let $\chi \in \emptyset$ be arbitrary.  Then every bounded equation is connected, multiply extrinsic and Riemannian.
\end{lemma}

\noindent You may also wish to include proofs. Sometimes, if these are long and nasty, they might be better off in the Appendix. For example, Appendix~\ref{AppendixA} illustrates the use of the proof environment.

\section{Algorithms}
You may also find it necessary to use pseudocode to describe some algorithm. This can be handled using the \texttt{algorithm} environment which gets to have a placement indicator and a caption like the figures and tables. Algorithm~\ref{Alg:1} gives an example of how this could be used~\cite{Huang1997} .

\begin{algorithm}[t]
\caption{Accumulate the co-occurrence matrix $S_\mathbf{d}(\theta, \Delta \theta)$, locating the junction position $\mathbf{x}$ and edge segment orientations $\{\theta_n\}_{n = 1}^N$ \label{Alg:1}}
\label{Alg:Accumulation}
\begin{algorithmic}
\State \textbf{Input:} Image I, centre of family of circles $\mathbf{c}$, size of family of circles $M$
\State \textbf{Output:} Co-occurrence matrix $S_{\mathbf{d}}(\theta, \Delta \theta)$\\

\ForAll{$\mathbf{d}$}

    \ForAll{$(\theta, \Delta \theta)$ pairs}

        \ForAll{$r$ in the family of circles}\\

            \State $\mathbf{k}_{\beta} = [\cos(\beta), \sin(\beta)]^T$
            \State $I_r(\beta) = \mathbf{c} + r\mathbf{k}_{\beta}$\\

            \State $\beta(\theta) = \theta \pm \sin^{-1}(\sfrac{d}{r}\sin(\alpha - \theta))$\\

            \State $S_{\mathbf{d}}(\theta, \Delta \theta) \gets S_{\mathbf{d}}(\theta, \Delta \theta) + \sfrac{1}{M}I_r(\beta(\theta))I_r(\beta(\theta + \Delta \theta))$\\
        \EndFor
    \EndFor
\EndFor\\

\State $(\hat{\mathbf{d}}, \hat{\theta}, \Delta \hat{\theta}) \gets \max\{S_{\mathbf{d}}(\theta, \Delta \theta)> T\}$\\
\State $\mathbf{x} \gets \mathbf{c} + \hat{\mathbf{d}}$
\State $\{\theta_n\}_{n = 1}^{N} \gets \bigcup\left\{\hat{\theta}, \hat{\theta} + \Delta \hat{\theta}\right\}$
\end{algorithmic}
\end{algorithm}
