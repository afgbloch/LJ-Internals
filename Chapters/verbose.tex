%!TEX root = ../FYP_Dissertation.tex
This module shows verbose information about the progress of the
JIT compiler. It prints one line for each generated trace. This module
is useful to see which code has been compiled or where the compiler
punts and falls back to the interpreter.

Example usage:

\begin{lstlisting}
  luajit -jv -e "for i=1,1000 do for j=1,1000 do end end"
  luajit -jv=myapp.out myapp.lua
\end{lstlisting}
To redirect the output to a file, pass a
filename as an argument (use '-' for stdout) or set the environment
variable LUAJIT\_VERBOSEFILE. The file is overwritten every time the
module is started.

The output from the second example could look like this:

\begin{center}
[TRACE   1 myapp.lua:1 loop]

[TRACE   2 (1/3) myapp.lua:1 $->$ 1]
\end{center}

The first number in each line is the internal trace number. Next are
the file name ('myapp.lua') and the line number (':1') where the
trace has started. Side traces also show the parent trace number and
the exit number where they are attached to in parentheses ('(1/3)').
An arrow at the end shows where the trace links to ('$->$ 1'), unless
it loops to itself.

In this case the inner loop gets hot and is traced first, generating
a root trace. Then the last exit from the 1st trace gets hot, too,
and triggers generation of the 2nd trace. The side trace follows the
path along the outer loop and \textit{around} the inner loop, back to its
start, and then links to the 1st trace.

Aborted traces are shown like this:
\begin{center}
[TRACE --- foo.lua:44 -- leaving loop in root trace at foo:lua:50]
\end{center}

Trace aborts are quite common, even in programs which
can be fully compiled. The compiler may retry several times until it
finds a suitable trace. This doesn't work with features that are
not-yet-implemented (NYI error messages). The VM simply falls back to the
interpreter. This may not matter at all if the particular trace is not very high
up in the CPU usage profile, plus the interpreter is quite fast, too.