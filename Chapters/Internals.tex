%!TEX root = ../FYP_Dissertation.tex

\section{String Internalization}
\label{Sec:string-inter}

All strings that are manipulated by LuaJIT are internalized. This includes, all
the strings literals of the user lua code, all identifier and tokens of the lua
code itself and also all the strings used internally by LuaJIT. Internalization
mechanism does that, only one copy of a specific strings is kept in memory. If
multiple copy of a same string are requested, a pointer to the internalized
version of the string is returned, instead of doing a new string allocation.
Strings need to be immutable and are zero-terminated. Strings function are
implemented in the \emph{lj\_str.c} file and internalization is done by the
\emph{lj\_str\_new} function.

For that, it implements a hash table and use a very sparse but fast hash
function. Collisions are handled by the use of singly-chained linked list.
The table is resized and all string rehashed when a 100\% load is reached.
The necessary states are saved in the \emph{global\_State} structure in the
\emph{lj\_obj.h} file.

\begin{lstlisting}[style=CStyle]
typedef struct global_State {
  GCRef *strhash; /* String hash table (hash chain anchors). */
  MSize strmask;  /* String hash mask (size of hash table - 1). */
  MSize strnum;   /* Number of strings in hash table. */
  [...]
}
\end{lstlisting}


