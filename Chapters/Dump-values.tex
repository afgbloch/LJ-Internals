%!TEX root = ../FYP_Dissertation.tex

In this appendix, is presented possible values that can be shown in some dump
and their corresponding meaning.
\begin{table}
\centering
\begin{tabular}{|l|l|}
\hline
\multicolumn{1}{|c|}{value} & \multicolumn{1}{c|}{Meaning}\\\hline
none                        & Incomplete trace. No link, yet.\\\hline
root                        & Link to other root trace.\\\hline
loop                        & Loop to same trace.\\\hline
tail-recursion              & Tail-recursion.\\\hline
up-recursion                & Up-recursion.\\\hline
down-recursion              & Down-recursion.\\\hline
\multirow{2}{*}{interpreter}& Fallback to interpreter (stop a side trace record due\\
& to maximum reached see \emph{sidecheck:} in lj\_record.c\\\hline
return                      & Return to interpreter.\\\hline
stitch                      & race stitching.\\\hline
\end{tabular}
\caption{
  Possible value for [link] \\(See jit\_trlinkname and TraceLink in lib\_jit.c
  and lj\_jit.h)
}
\label{tab:dump-link}
\end{table}
\begin{table}
\centering
\begin{tabular}{|l|l|l|}
\hline
P & IRSLOAD\_PARENT    & Coalesce with parent trace.\\
F & IRSLOAD\_FRAME     & Load 32 bits of ftsz.\\
T & IRSLOAD\_TYPECHECK & Needs type check.\\
C & IRSLOAD\_CONVERT   & Number to integer conversion.\\
R & IRSLOAD\_READONLY  & Read-only, omit slot store.\\
I & IRSLOAD\_INHERIT   & Inherited by exits/side traces.\\
\hline
\end{tabular}
\caption{
  Possible value for SLOAD argument \\(See lj\_ir.h)
}
\label{tab:dump-sload}
\end{table}
\begin{table}
\centering
\begin{tabular}{|l|l|l|}
\hline
R & IRXLOAD\_READONLY  & Load from read-only data.\\
V & IRXLOAD\_VOLATILE  & Load from volatile data.\\
U & IRXLOAD\_UNALIGNED & Unaligned load.\\
\hline
\end{tabular}
\caption{
  Possible value for XLOAD argument \\(See lj\_ir.h)
}
\label{tab:dump-sload}
\end{table}
\begin{table}
\centering
\begin{tabular}{|l|l|l|}
\hline
str.len        & IRFL\_STR\_LEN        & \\
func.env       & IRFL\_FUNC\_ENV       & \\
func.pc        & IRFL\_FUNC\_PC        & \\
func.ffid      & IRFL\_FUNC\_FFID      & Function id\\
thread.env     & IRFL\_THREAD\_ENV     & \\
tab.meta       & IRFL\_TAB\_META       & \\
tab.array      & IRFL\_TAB\_ARRAY      & \\
tab.node       & IRFL\_TAB\_NODE       & Hash part\\
tab.asize      & IRFL\_TAB\_ASIZE      & Size of array part\\
tab.hmask      & IRFL\_TAB\_HMASK      & Size of hash part - 1\\
\multirow{3}{*}{tab.nomm} & \multirow{3}{*}{IRFL\_TAB\_NOMM} & Negative cache for fast \\
& & metamethods bitmap, marking\\
& & absent fields of the metatable\\
udata.meta     & IRFL\_UDATA\_META     & \\
udata.udtype   & IRFL\_UDATA\_UDTYPE   & See UDTYPE table\\
udata.file     & IRFL\_UDATA\_FILE     & udata payload \\
cdata.ctypeid  & IRFL\_CDATA\_CTYPEID  & \\
cdata.ptr      & IRFL\_CDATA\_PTR      & cdata payload \\
cdata.int      & IRFL\_CDATA\_INT      & cdata payload \\
cdata.int64    & IRFL\_CDATA\_INT64    & cdata payload \\
cdata.int64\_4 & IRFL\_CDATA\_INT64\_4 & cdata payload \\
\hline
\end{tabular}
\caption{
  Possible value for FLOAD or FREF argument \\(See lj\_ir.h)
}
\label{tab:dump-fload-fref}
\end{table}
\begin{table}
\centering
\begin{tabular}{|l|l|}
\hline
UDTYPE\_USERDATA  & Regular userdata.\\
UDTYPE\_IO\_FILE  & I/O library FILE.\\
UDTYPE\_FFI\_CLIB & FFI C library namespace.\\\hline
\end{tabular}
\caption{
  Possible value for userdata types \\(See lj\_obj.h)
}
\label{tab:dump-fload-fref}
\end{table}
\begin{table}
\centering
\begin{tabular}{|l|l|}
\hline
floor & FPM\_FLOOR \\
ceil  & FPM\_CEIL  \\
trunc & FPM\_TRUNC \\
sqrt  & FPM\_SQRT  \\
exp   & FPM\_EXP   \\
exp2  & FPM\_EXP2  \\
log   & FPM\_LOG   \\
log2  & FPM\_LOG2  \\
log10 & FPM\_LOG10 \\
sin   & FPM\_SIN   \\
cos   & FPM\_COS   \\
tan   & FPM\_TAN   \\
\hline
\end{tabular}
\caption{
  Possible value for FPMATH argument \\(See lj\_ir.h)
}
\label{tab:dump-fpmath}
\end{table}
\begin{table}
\centering
\begin{tabular}{|l|l|}
\hline
IRBUFHDR\_RESET  & Reset buffer \\
IRBUFHDR\_APPEND & Append to buffer \\
\hline
\end{tabular}
\caption{
  Possible value for BUFHDR argument \\(See lj\_ir.h)
}
\label{tab:dump-bufhdr}
\end{table}

\begin{table}
\centering
\begin{tabular}{|l|l|l|}
\hline
INT  & IRTOSTR\_INT  & Convert integer to string.  \\
NUM  & IRTOSTR\_NUM  & Convert number to string.  \\
CHAR & IRTOSTR\_CHAR & Convert char value to string.  \\\hline
\end{tabular}
\caption{
  Possible value for TOSTR argument \\(See lj\_ir.h)
}
\label{tab:dump-tostr}
\end{table}