%!TEX root = ../FYP_Dissertation.tex
\section{Hotpath detection}
\label{Sec:Hotpath}
LuaJIT does trace compilation. For that, it needs to detect that a certain
portion of the code gets hot and compile it as a trace. Two types of trace entry
are detected, loops and function. In the vm (See Part \ref{Part:VM}) each
traceable loop-like or call-like BC (byte-code) decrement a counter that is the
hashed pc (program counter) position of a (64-entries) table. When the counter underflows it start recording. At the end of the recording phase, it patch the
BC with a call to the compiled trace.

\paratitle{Loops:}

\begin{itemize}
	\item \emph{hotloop} : hash the pc, decrement the corresponding counter and
jmp to vm\_hotloop if it underflows (vm\_(arch).dasc).
	\item \emph{vm\_hotloop} : Prepare the stack and call lj\_trace\_hot.
	\item \emph{lj\_trace\_hot} : start recording a trace (lj\_trace.c).
\end{itemize}

\paratitle{Functions:}

\begin{itemize}
	\item \emph{hotcall} : hash the pc, decrement the corresponding counter and
jmp to vm\_hotcall if it underflows (vm\_(arch).dasc).
	\item \emph{vm\_hotcall} : Prepare the stack and call lj\_dispatch\_call.
	\item \emph{lj\_dispatch\_call} : start recording a trace (lj\_dispatch.c).
\end{itemize}