%!TEX root = ../FYP_Dissertation.tex
This module can be used to debug the JIT compiler itself. It dumps the
code representations and structures used in various compiler stages.

Example usage:
\begin{lstlisting}
  luajit -jdump -e "
    local x=0
    for i=1,1e6 do x=x+i end
    print(x)
  "
  luajit -jdump=im -e "
    for i=1,1000 do
      for j=1,1000 do end
    end
  " | less -R
  luajit -jdump=is myapp.lua | less -R
  luajit -jdump=-b myapp.lua
  luajit -jdump=+aH,myapp.html myapp.lua
  luajit -jdump=ixT,myapp.dump myapp.lua
\end{lstlisting}
The first argument specifies the dump mode. The second argument gives
the output file name. Default output is to stdout, unless the environment
variable LUAJIT\_DUMPFILE is set. The file is overwritten every time the
module is started. Different features can be turned on or off with the dump mode.
If the mode starts with a '+', the following features are added to the default
set of features; a '-' removes them. Otherwise the features are replaced.\\
The following dump features are available (* marks the default):

\begin{itemize}
  \item \textbf{t *} - Print a line for each started, ended or aborted trace (see also -jv).
  \item \textbf{b *} - Dump the traced bytecode.
  \item \textbf{i *} - Dump the IR (intermediate representation).
  \item \textbf{r} - Augment the IR with register/stack slots.
  \item \textbf{s} - Dump the snapshot map.
  \item \textbf{m *} - Dump the generated machine code.
  \item \textbf{x} - Print each taken trace exit.
  \item \textbf{X} - Print each taken trace exit and the contents of all registers.
  \item \textbf{a} - Print the IR of aborted traces, too.
\end{itemize}
The output format can be set with the following characters:
\begin{itemize}
   \item \textbf{T} - Plain text output.
   \item \textbf{A} - ANSI-colored text output
   \item \textbf{H} - Colorized HTML + CSS output.
\end{itemize}
The default output format is plain text. It's set to ANSI-colored text
if the COLORTERM variable is set.