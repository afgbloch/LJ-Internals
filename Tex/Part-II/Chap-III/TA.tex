%!TEX root = ../../../FYP_Dissertation.tex

When the recorder state machine reaches the \emph{LJ\_TRACE\_ASM} state, the
\emph{lj\_asm\_trace} main assembler function is called. This function is
responsible for the setup and teardown of the assembler phase. Each IR
instruction of the trace is assembled through the \emph{asm\_ir} function which
is a switch case on the opcode. The implementation of the assembler is divided
in three different files. \emph{lj\_asm.c} contains the assembler code that is
agnostic of the platform, \emph{lj\_asm\_(target).h} contains the assembler code
specific to a particular target (i.e. x86), and \emph{lj\_emit\_(target).h}
contain the helper functions for generating instructions for a specific
instruction-set. A trace is assembled in linear backwards order. It uses
\emph{ASMState} has the main data structure that helps with, physical register
allocation, machine code and IR traversal, and snapshots handling. At the end of
the assembler phase, the bytecode is patched to use instead of the interpreted
bytecode a call to the compiled trace.

% -- TO BE ADDED ????
% in lj_trace.c ? (verify)
% - lj_record_stop
%   - trace stitching (call to c using c API)
%   - end of the hotloop
%   - hit an other already compiled loop (link to it)
%   - return statement
%     - return to interpretor for unhandled cases
%     - when downrec limit reach for side-trace (nagative frame depth)
%     - tail-rec is detected (limit reach in the same framedepth)
%     - limit reach in up recursion (positive frame depth)
%   - hit a compiled function (link to its trace)