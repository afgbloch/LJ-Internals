%!TEX root = ../../../FYP_Dissertation.tex

%===============================================================================
% Tagged value
%===============================================================================

\subsection{Tagged value}
\label{Subsec:tagged-value}

LuaJIT represent all internal elements as 64-bits \emph{TValue} (tagged value).
It uses the nan-tagging technique to differentiate between numbers and other
types of element. In fact, \emph{lua\_number} are 64-bits floating-points numbers
following the \emph{ieee} standard. Doing so, numeric nan are canonized by the
cpu FPU (0xfff8 in msb and zero otherwise), letting the possibility to use the
lower bits to represent arbitrary data. Internally, LuaJIT has two different
representations, one for x86 and another for x64 bit (\emph{LJ\_GC64}) mode
(see Tables \ref{tab:tagged-gc32} and \ref{tab:tagged-gc64}). In those tables,
\emph{itypes} are numbers identifying the type of the object.
GC object (Garbage Collectible Object) represent all allocated object
that are managed by the garbage collector. \emph{GCRef} are references to such
object.

\begin{table}[H]
\centering
\caption{Internal object tagging for 32-bits mode}
\label{tab:tagged-gc32}
\begin{tabular}{l|c|c|c|}
\cline{2-4}
                                              & \multicolumn{2}{c|}{MSW}         & LSW         \\ \hline
\multicolumn{1}{|l|}{size}                    & \multicolumn{2}{c|}{32-bits}     & 32-bits     \\ \hline
\multicolumn{1}{|l|}{primitive types}         & \multicolumn{2}{c|}{itypes}      &             \\
\multicolumn{1}{|l|}{lightuserdata (32-bits)} & \multicolumn{2}{c|}{itypes}      & void *      \\
\multicolumn{1}{|l|}{lightuserdata (64-bits)} & 0xffff           & \multicolumn{2}{c|}{void *} \\
\multicolumn{1}{|l|}{GC objects}              & \multicolumn{2}{c|}{itypes}      & GCRef       \\
\multicolumn{1}{|l|}{int}                     & \multicolumn{2}{c|}{itypes}      & int         \\
\multicolumn{1}{|l|}{number}                  & \multicolumn{3}{c|}{double}                    \\ \hline
\end{tabular}
\end{table}

\begin{table}[H]
\centering
\caption{Internal object tagging for 64-bits mode}
\label{tab:tagged-gc64}
\begin{tabular}{l|c|c|c|c|}
\cline{2-5}
                                      & \multicolumn{3}{c|}{MSW}         & LSW         \\ \hline
\multicolumn{1}{|l|}{size}            & 13-bits & 4-bits & 15-bits       & 32-bits     \\ \hline
\multicolumn{1}{|l|}{primitive types} & 1...1   & itype  & \multicolumn{2}{c|}{1...1}  \\
\multicolumn{1}{|l|}{lightuserdata}   & 1...1   & itype  & \multicolumn{2}{c|}{void *} \\
\multicolumn{1}{|l|}{GC objects}      & 1...1   & itype  & \multicolumn{2}{c|}{GCRef}  \\
\multicolumn{1}{|l|}{int}             & 1...1   & itype  & 0...0         & int         \\
\multicolumn{1}{|l|}{number}          & \multicolumn{4}{c|}{double}                    \\ \hline
\end{tabular}
\end{table}

%===============================================================================
% String Internalization
%===============================================================================

\subsection{String Internalization}
\label{Subsec:string-inter}

All strings that are manipulated by LuaJIT are internalized. This includes, all
the strings literals of the user lua code, all identifiers and tokens of the lua
code itself and also all the strings used internally by LuaJIT. Internalization
mechanism does that, only one copy of a specific string is kept in memory. If
multiple copies of a same string is requested, a pointer to the internalized
version of the string is returned, instead of doing a new string allocation.
Strings need to be immutable and are zero-terminated. Strings function are
implemented in the \emph{lj\_str.c} file and internalization is done by the
\emph{lj\_str\_new} function.

For that, it implements a hash table and use a very sparse but fast hash
function. Collisions are handled by the use of singly-chained linked list.
The table is resized and all string rehashed when a 100\% load is reached.
The necessary states are saved in the \emph{global\_State} structure in the
\emph{lj\_obj.h} file.

\begin{lstlisting}[style=CStyle]
typedef struct global_State {
  GCRef *strhash; /* String hash table (hash chain anchors). */
  MSize strmask;  /* String hash mask (size of hash table - 1). */
  MSize strnum;   /* Number of strings in hash table. */
  [...]
}
\end{lstlisting}

%===============================================================================
% Lua table
%===============================================================================

\subsection{Lua table}
\label{Subsec:table}

Tables are garbage collectible objects represented by the structure below
(\emph{GCtab} in \emph{lj\_obj.h}). Functions to manipulate them are in
\emph{lj\_tab.c} and \emph{lj\_tab.h} files. It is composed of an array part and
a hash part. If the array part is small, it is allocated directly after the
structure in memory (collocation functionality), otherwise it is separated. The
hash part is a hash table used to store all non-integer key (or integer too big
to fit in the array part). It is implemented has an array using singly-linked
list for collision, where nodes of the linked list are within the array (not
allocated) and a variation of \emph{Brent's hashing methods} is used.
New integer keys that are bigger than the array part are always inserted
in the hash part until this one is full. It then triggers the resizing of the
table. The new \emph{asize} and \emph{hmask} are both power of 2. The
new \emph{asize} value corresponds to the biggest power of 2 such that at least
50 percent of the integers below it are used as key. The new \emph{hmask} is
picked such that all non-integer keys plus the integer keys that are bigger than
the new \emph{asize} fit in it. When a resizing occur the hash values are
re-hashed and integer key that does fit now in the array part are reintroduced
there.

The \emph{nomm} field of GCtab is a
negative cache for fast metamethods checks. It is a bitmap marking absent fields
of the metatable.

\begin{lstlisting}[style=CStyle]
typedef struct GCtab {
  GCHeader;
  uint8_t nomm;    /* Negative cache for fast metamethods. */
  int8_t colo;     /* Array colocation
  (number of slot directly following in memory).*/
  MRef array;      /* Array part. */
  GCRef gclist;
  GCRef metatable; /* Must be at same offset in GCudata. */
  MRef node;       /* Hash part. */
  uint32_t asize;  /* Size of array part (keys [0, asize-1]). */
  uint32_t hmask;  /* Hash part mask (size of hash part - 1). */
#if LJ_GC64
  MRef freetop;    /* Top of free elements. */
#endif
} GCtab;
\end{lstlisting}

%===============================================================================
% Garbage collector
%===============================================================================

\subsection{Garbage collector}
\label{Subsec:gc}

LuaJIT garbage collector has currently a tricolor, incremental mark and sweep
type implementation. You can find a presentation of it on the wiki
\cite{luajit-gc} in the tri-color section. Its source code can be found in the
\emph{lj\_gc.h} and \emph{lj\_gc.c} files. It uses the GCState structure has a
principal structure. The \emph{gc\_onestep} function is the principal one as it
implements the states machine of the gc. Principal states with their meaning can be
found in the enum below.

\begin{lstlisting}[style=CStyle]
enum {
  GCSpause,
  GCSpropagate,   /* One gray object is processed. */
  GCSatomic,      /* Atomic transition from mark to sweep phase. */
  GCSsweepstring, /* Sweep one chain of strings from the table. */
  GCSsweep,       /* Sweep a few object from root. */
  GCSfinalize     /* Finalize one userdata or cdata object. */
};
\end{lstlisting}

\begin{lstlisting}[style=CStyle]
typedef struct GCState {
  GCSize total;         /* Memory currently allocated. */
  GCSize threshold;     /* Memory threshold. */
  uint8_t currentwhite; /* Current white color. */
  uint8_t state;        /* GC state. */
  uint8_t nocdatafin;   /* No cdata finalizer called. */
  uint8_t unused2;
  MSize sweepstr;       /* Sweep position in string table. */
  GCRef root;           /* List of all collectable objects. */
  MRef sweep;           /* Sweep position in root list. */
  GCRef gray;           /* List of gray objects. */
  GCRef grayagain;      /* List of objects for atomic traversal. */
  GCRef weak;           /* List of weak tables (to be cleared). */
  GCRef mmudata;        /* List of userdata (to be finalized). */
  GCSize debt;          /* Debt (how much GC is behind schedule). */
  GCSize estimate;      /* Estimate of memory actually in use. */
  MSize stepmul;        /* Incremental GC step granularity. */
  MSize pause;          /* Pause between successive GC cycles. */
} GCState;
\end{lstlisting}

%===============================================================================
% Allocator
%===============================================================================

\subsection{Allocator}
\label{Subsec:alloc}

LuaJIT has its own embedded allocator that is a customized version of \emph{dlmalloc}
(Doug Lea's Malloc). Information on the original implementation can be found on
the web article \cite{dlmalloc-art} or in the comment of the code
\cite{dlmalloc-impl}. The allocator is implemented in the \emph{lj\_alloc.c}
file. Its main structure \emph{malloc\_state} is shown below. Memory on the
heap are allocated in chunks. Free chunks are managed as double linked-list with
the size of the chunk at the beginning and end of it. Unallocated memory are
grouped in bins of corresponding size. There are two types of bins. The
smaller one contains chunk of same size and the top are anchored in
\emph{smallbins}. The bigger one are stored as bitwise digital trees (aka tries)
keyed by size were the top of a tree is anchored in \emph{treebins}. The
allocator differentiates with two types of memory allocation, if it is higher than
128KB then it asks the operating system for a new memory segment using mmap, if
it is lower than that it uses chunks from the current segment. All allocated
segments are kept in a linked list anchored in \emph{seg}. For such smaller
allocation the allocator first try to find an exact fit from the available
chunks to optimize for internal fragmentation. If it cannot find one and that
the requested size is smaller than \emph{dvsize} then it uses the \emph{dv}
chunk (designated victim) which is the last chunk that has been split. It does
so to optimize for locality. Otherwise it goes for a best-fit match. If no chunk
big enough is available, it asks the system to extend the segment and use the
boundary chunk \emph{top} (always kept free). When memory is freed it does chunk
coalescing to avoid memory fragmentation. If \emph{topsize} is bigger than
\emph{trim\_check}, then the current segment is shrinked and the memory is given
back to the OS. \emph{release\_checks} is a decreasing counter that when it
riches zero triggers a check of all segments to release empty ones back to the OS.

\begin{lstlisting}[style=CStyle]
struct malloc_state {
  binmap_t  smallmap;       /* marking non-empty smallbins */
  binmap_t  treemap;        /* marking non-empty treebins  */
  size_t    dvsize;         /* designated victim size      */
  size_t    topsize;        /* boundary chunk size         */
  mchunkptr dv;             /* designated victim           */
  mchunkptr top;            /* boundary chunk              */
  size_t    trim_check;     /* threshold size to release   */
  size_t    release_checks; /* counter for mmap release    */
  mchunkptr smallbins[...]; /* anchor for top small bins   */
  tbinptr   treebins[...];  /* anchor for top tree bins    */
  msegment  seg;            /* anchor for chained segments */
};
\end{lstlisting}

%===============================================================================
% Function
%===============================================================================

\subsection{Function}
\label{Subsec:func}

There are two distinct representation for function, the function's prototype,
and the function's closure. the Lua functions' prototypes are represented by \emph{GCproto} (\emph{lj\_obj.h}), and are followed by the functions' bytecodes in memory.
The closures are represented by the GCfuncL for lua function and GCfuncC for
c function (using lua's api). They contain the necessary information for
upvalues. Upvalues are represented by the \emph{GCupval} that if close, contains
the corresponding value or a reference to the stack slot containing the
appropriate value otherwise. Closure can be managed by the functions present in
\emph{lj\_func.c} allowing to create them, destroy them and closing their
upvalues.

\begin{lstlisting}[style=CStyle]
typedef struct GCproto {
  GCHeader;
  uint8_t numparams; /* Number of parameters. */
  uint8_t framesize; /* Fixed frame size. */
  MSize sizebc;      /* Number of bytecode instructions. */
  [...]
  GCRef gclist;
  MRef k;            /* Split constant array (ptr to the middle).*/
  MRef uv;           /* Upvalue list. local slot or parent uv idx. */
  MSize sizekgc;     /* Number of collectable constants. */
  MSize sizekn;      /* Number of lua_Number constants. */
  MSize sizept;      /* Total size including colocated arrays. */
  uint8_t sizeuv;    /* Number of upvalues. */
  uint8_t flags;     /* Miscellaneous flags. */
  uint16_t trace;    /* Anchor for chain of root traces. */
  [...]
} GCproto;
\end{lstlisting}

%===============================================================================
% Fast Function
%===============================================================================

\subsection{Fast Function}
\label{Subsec:ffunc}

Fast functions are specially optimized standard library function. There are two
parts for each function, the implementation part and the fallback handler.
The implementation part is done in assembly in the \emph{vm\_*.dasc} files and
handle the general case, called the fast pass. If the fast pass fail, the vm
call the corresponding fallback handler (all \emph{LJLIB\_ASM} marked functions)
that will try to recover from the failure if possible. Examples of recoverable
cases might be : a wrong argument type if coercion succeeds or a stack overflow
if stack reallocation succeeds (see comment in lj\_lib.h).
