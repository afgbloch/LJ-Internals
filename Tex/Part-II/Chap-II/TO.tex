%!TEX root = ../../../FYP_Dissertation.tex

We can differentiate in the code, three types of optimizations.
First of all, there are the optimizations present in the optimization engine.
They are implemented in the \emph{lj\_opt\_*.c} files and thus easy to spot out.
Those optimization are of two kinds, global optimization that are run on the
entire IR at once at the end of the recording phase, during the
\emph{LJ\_TRACE\_END} state (See \ref{Subsec:opt-dce}, \ref{Subsec:opt-loop},
\ref{Subsec:opt-split}, \ref{Subsec:opt-sinking}) and the local optimizations
that are applied while recording a trace (See \ref{Subsec:narrowing},
\ref{Subsubsec:fold}, \ref{Subsubsec:mao}). Finally, there is a plethora of
optimization and heuristic applied here and there (See LuaJIT wiki on
optimization \cite{luajit-opt}).

%===============================================================================
% Dead code elimination
%===============================================================================

\subsection{Dead code elimination}
\label{Subsec:opt-dce}

DCE or Dead Code Elimination is performed by the lj\_opt\_dce main function
in two phases. During the first phase called mark snap, it mark all IR
instructions that are referenced by a snapshot. The second phases called
propagate, iteratively mark all IR instruction that are referenced by an already
marked IR instruction while replacing non-marked IR instruction by nops.

%===============================================================================
% Loop optimizations
%===============================================================================

\subsection{Loop optimizations}
\label{Subsec:opt-loop}

The loop optimization is a way to improve code hoisting for trace based on
loops. In fact LuaJIT should try to hoist most invariant instruction and guards
in such a way that trace that doesn't match the current dynamic profile of
the code (assumption on data or type) are exited as soon as possible.
Unfortunately due to the dynamic nature of the IR generated, it contains many
guards and thus are control-dependent, making little room for loop-invariant
code motion (LICM). The solution used here is a copy-substitution of the body
of the loop. It basically consists in always unrolling the loop once before the
actual loop instruction. This allows the code executed inside the loop to
contains only variant instructions.

%===============================================================================
% Split optimizations
%===============================================================================

\subsection{Split optimizations}
\label{Subsec:opt-split}

The split optimization is only used by 32-bits architecture and splits the
64-bits IR instructions into multiple 32-bits once.

%===============================================================================
% Sinking optimizations
%===============================================================================

\subsection{Sinking optimizations}
\label{Subsec:opt-sinking}
This is a very useful optimization that allows to avoid many temporaries and
unnecessary memory accesses and allocations by keeping the values of interest
directly in register. Since memory modifications are not performed, we need a way
to remember the writes in case those value escape the execution path
(are not temporary anymore). For this purpose snapshots are used (See Section \ref{Subsec:snap}).
This optimization is implemented in the \emph{lj\_opt\_sink.c} file. A detailed
explanation of this optimization is available on the wiki \cite{luajit-sink}.

%===============================================================================
% Narrowing optimizations
%===============================================================================

\subsection{Narrowing optimizations}
\label{Subsec:narrowing}

It performs the narrowing of Lua numbers (double) into integers when it seems
to be useful. LuaJIT use demand-driven (by the backend) narrowing for index
expressions, integer arguments (FFI), and bit operations and predictive
narrowing for induction variables. It emits overflow check instruction when
necessary. Most arithmetic operations are never narrowed. To learn more on why,
see the comment section in the \emph{lj\_opt\_narrow.c} file.

%===============================================================================
% Fold engine
%===============================================================================

\subsection{Fold engine}
\label{Subsec:fold}

The fold engine implement a rule-based mechanism. Rules are declared using the
LJFOLD macro which contains the IR opcode and a rule on the parameters it
applies to. During the LuaJIT buildvm (more precisely the \emph{buildvm\_fold.c}
file) those rules are scanned and the \emph{lj\_foldef.h} file gets generated.
It contains a semi-perfect hash table for constant-time rule lookups, where each
entry respect the format depicted in Table \ref{tab:fold-format}. It also
contains a second table containing the function to call if a corresponding rule
match.

\begin{table}[H]
\centering
\caption{Fold hash table, bit pattern entry}
\label{tab:fold-format}
\begin{tabular}{|c|c|c|c|c|}
\hline
8 bits           & 7 bits            & 7 bits            & 2 bits   & 8 bits                      \\ \hline
index fold table & fold instr opcode & left instr opcode & 00       & right instr opcode      \\ \hline
index fold table & fold instr opcode & left instr opcode & \multicolumn{2}{c|}{literal field} \\ \hline
\end{tabular}
\end{table}

\subsubsection{Fold optimizations}
\label{Subsubsec:fold}

The fold optimizations preformed by the FOLD engine are implemented in the
\emph{lj\_opt\_fold.c} file and can be separated in five well-known techniques,
constant folding, algebraic simplifications, reassociation, common subexpression
elimination and array bounds check elimination.

\subsubsection{Memory access optimizations}
\label{Subsubsec:mao}

The memory access optimization perform by the FOLD engine and implemented
in the lj\_opt\_mem.c file consists of three components, the alias
analysis using high-level semantic disambiguation, Load to load and store to load
forwarding, and finally dead-store elimination.
