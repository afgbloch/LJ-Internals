%!TEX root = ../../../FYP_Dissertation.tex

%===============================================================================
% Standard library
%===============================================================================

\subsection{Standard library}
\label{Subsec:std-lib}

LuaJIT is providing a full compatibility with Lua 5.1 and to do so, it
implements the standard library. The code is copied and adapted from the
\emph{PUC-RIO} Lua interpreter. A list of the corresponding files and the
descriptions of what they are doing can be found in table
\ref{tab:library-std-files}.

\begin{table}[H]
\centering
\caption{Implementation's files of the Lua standard library}
\label{tab:library-std-files}
\begin{tabular}{|l|l|}
\hline
\multicolumn{1}{|c|}{File name} & \multicolumn{1}{c|}{Description}                     \\ \hline
lib\_base.c                     & Base and coroutine library.                          \\
lib\_debug.c                    & Debug library.                                       \\
lib\_init.c                     & Load and initialize standard libraries.              \\
lib\_io.c                       & Files and I/O library.                               \\
lib\_math.c                     & Math library (abs, sqrt, log, random, etc...).       \\
lib\_os.c                       & OS library (date, time, execute, remove, etc...).    \\
lib\_package.c                  & Package library (load, require, etc...).             \\
lib\_string.c                   & String library (gsub, match, etc...).                \\
lib\_table.c                    & Table library (new, clear, insert, foreach, etc...). \\ \hline
\end{tabular}
\end{table}

%===============================================================================
% LuaJIT extensions
%===============================================================================

\subsection{LuaJIT extensions}
\label{Subsec:lj-extensions}

In addition to the standard library, LuaJIT comes equipped with some library
extensions \cite{extensions}. In addition of a few improvements of existing
modules it provides three new extension modules, namely, \emph{bit} in
\emph{lib\_bit.c} that provides with bitwise operations \cite{bitOp}, \emph{ffi}
in \emph{lib\_ffi.c} that provides functions to interact with the FFI library
(see FFI Chapter \ref{Sec:FFI} for more) and \emph{JIT} in \emph{lib\_jit.c}
that provides functions allowing to control the behavior of the JIT compiler
engine (see JIT Part \ref{Chapt:JIT}).

%===============================================================================
% The C API
%===============================================================================

\subsection{The C API}
\label{Subsec:c-api}

All those libraries are implemented using the Lua C API that allows to create
and manipulate Lua data, manage the Lua stack etc... The implementation for those
functions can be found in \emph{lj\_api.c}, \emph{lj\_lib.c} and \emph{lib\_aux.c}.

%===============================================================================
% Build Library
%===============================================================================

\subsection{Build Library}
\label{Subsec:build-lib}

LuaJIT use multiple tricks to generate files automatically that are included
during compilation to help with the building and loading of the standard library
without the need for manual maintenance. We will describe here the different
steps and what they are useful for.

First of all, if no Lua interpreter
(either PUC-Lua or LuaJIT) is available on the machine a simplified and
minimized one is built from the \emph{minilua.c} file. Then the interpreter is
used to run \emph{genlibbc.lua} that will be responsible for parsing all LuaJIT's
source files searching for the \emph{LJLIB\_LUA} macro that surrounds library
functions name that are written in Lua. It then generates the \emph{buildbm\_libbc.h}
file that contains the Lua bytecodes for all those functions in the
\emph{libbc\_code} array and a mapping of the function name and the bytecodes
offset for that function in \emph{libbc\_map}.

This newly generated file is
built along with all \emph{buildvm\_*} files to create the \emph{buildvm}
program that is used to parse from the library source code all other
\emph{LJLIB\_*} macro and generates some files
(\emph{lj\_bcdef.h, lj\_libdef.h, lj\_ffdef.h, lj\_recdef.h and vmdef.lua})
that will be added to LuaJIT compilation. You can see, in Table
\ref{tab:library-macro} the description of the macros and in Table
\ref{tab:library-generated-files} the description of the corresponding generated
file.

\begin{table}[H]
\centering
\caption{Definition of the macros used to build the library}
\label{tab:library-macro}
\begin{tabularx}{\textwidth}{|l|X|}
\hline
\multicolumn{1}{|c|}{Macro}          & \multicolumn{1}{c|}{Description}                     \\\hline
LJLIB\_MODULE\_*                     & register new module.                                 \\\hline
LJLIB\_CF(name)                      & register C function.                                 \\\hline
LJLIB\_ASM(name)                     & register fast function fallback handler.             \\\hline
LJLIB\_ASM\_(name)                   &
  \begin{tabular}[c]{@{}l@{}}
  register fast function that uses previous\\
  LJLIB\_ASM fallback handler.
  \end{tabular}                                                                             \\\hline
LJLIB\_LUA(name)                     & register Lua function.                               \\\hline
\multirow{4}{*}{LJLIB\_SET(name)}    &
  \begin{tabular}[c]{@{}l@{}}
  register previous Lua stack value into the module\\
  table with \emph{name} has key.
  \end{tabular}
  \begin{itemize}
  \item '!' : last stack value became next function's env
  \end{itemize}                                                                             \\\hline
\multirow{7}{*}{LJLIB\_PUSH(val)}    & push \emph{val} on the Lua stask.
  \begin{itemize}
  \item 'lastcl'  : copy last stack value
  \item 'top-x'   : copy last $x^{th}$ stack value
  \item ' "..." ' : push internalized string
  \end{itemize}                                                                             \\\hline
\multirow{8}{*}{LJLIB\_REC(handler)} & register a handler to record a function.
  \begin{itemize}
  \item '.' : get the function's name
  \item 'name data' :
    \begin{itemize}
      \item \emph{name} of recorder
      \item auxiliary \emph{data} to put in \emph{recff\_idmap}
    \end{itemize}
  \end{itemize}                                                                             \\\hline
LJLIB\_NOREGUV                       & to not register this function in module.             \\\hline
LJLIB\_NOREG                         & to not register a function in lj\_lib\_cf\_*.        \\\hline
\end{tabularx}
\end{table}

\begin{table}[H]
\centering
\caption{Generated files description}
\label{tab:library-generated-files}
\begin{tabularx}{\textwidth}{|l|X|}
\hline
\multicolumn{1}{|c|}{File}          & \multicolumn{1}{c|}{Description}                      \\\hline
\emph{lj\_bcdef.h}                  &
  for each fast functions, \emph{lj\_bc\_ofs} contains the offset from
  \emph{lj\_vm\_asm\_begin} (in \emph{lj\_vm.h}) to the mcode of the function
  and \emph{lj\_bc\_mode} contains the byte code operande mode (all set to
  \emph{BCMODE\_FF}) (see \emph{lj\_bc.h} and this \emph{Introduction} section
  of the wiki \cite{luajit-bc}).                                                            \\\hline
\emph{lj\_ffdef.h}                  & list of all library function name                     \\\hline
\emph{lj\_libdef.h}                 &
  \emph{lj\_lib\_cf\_*} arrays contains the list of function pointers for the
  * library. \emph{lj\_lib\_init\_*} are arrays of packed data describing how
  the corresponding library should be loaded (see \emph{lj\_lib\_register} in
  \emph{lj\_lib.c} for the function that parse those data).                                 \\\hline
\emph{lj\_recdef.h}                 &
  for each library functions \emph{recff\_idmap} contains an optional auxiliary
  data (opcode, literal) allowing to handle similar functionalities in a
  common handler. \emph{recff\_func} contains the list of record handler.                   \\\hline
\emph{vmdef.lua}                    &
  contains all vm definition for use in Lua.
  \begin{itemize}
  \item bcnames  : bytecode names
  \item irnames  : IR instructions names
  \item irfpm    : floating point math function names
  \item irfield  : maps field id to field name (specific field that the IR instruction is accessing)
  \item ircall   : name of special function for calls instruction.
  \item traceerr : maps error num to error message
  \item ffnames  : maps library function id to function name
  \end{itemize}                                                                             \\\hline
\end{tabularx}
\end{table}
