%!TEX root = ../../../FYP_Dissertation.tex

In this section we present another case of surprising performance of MAD that
ended up being also attributed to the object model. In this example we are going
to talk about the iterator of the \emph{Sequence} data-structure.

To recall,
\emph{Sequence} is used to represent particle accelerator by composition of
different Elements. The iterator returns, the current element, its index, and its \emph{spos}
(distance in meters from the start of the sequence). The
only difference between forward and backward iterators is that for forward iterator
the \emph{spos} is precomputed for each elements during the \emph{sequence} creation, while for
backward we need to take the \emph{spos} of the element and add its length (i.e. make a lookup).
The difference in performance between forward and backward relies only on the
access of the length of an element.

The code below shows the two examples that we are comparing. We iterate in each
case 20000 times over the \emph{LHCB1} sequence (a circular accelerator with 3000 elements) with
the first case using the forward iterator and the second case using the backward
iterator (iterate from the end of the sequence to the beginning).

The figure \ref{fig:MO-seq-iter-curve}, shows the performance of ten independent
runs of the two cases. The forward iterator take around 0.3 seconds when the
backward one takes around 30 seconds to complete. The first thing to do
to try to understand the problem is to study the trace dump. The figure \ref{fig:MO-seq-iter}
shows a screenshot of the generated dump files and we can clearly see that
something goes wrong as it generates a file 300 times larger in the backward case.

Indeed the simplest examples explored in Section \ref{Sec:MO-perf-analys} don't
show what happens when each object that we query stores the requested variable at
different hierarchy depths. This is exactly what we have here, the figure
\ref{fig:MO-seq-iter-schematic} illustrates how the different elements
constituting the sequence might look like. The requested \emph{l} value (the length
of the corresponding element) might be stored at a different depth depending on
the kind of the element. Since the hierarchical lookup is completely inlined
in the trace, we end up with many different set of traces for each
depth.

A potential fix for this particular case would be to also store the
backward \emph{spos} in a table during the \emph{Sequence} construction.

\begin{lstlisting}[style=LuaStyle]
-- load LHC 1st beam sequence
local lhcb1 = loadLHC()

local cnt = 0

-- 1st case : forward iterator
for i, elem, s in lhcb1:iter(nil, 2e4, 1) do
	cnt = cnt + i
end

-- 2nd case : backward iterator (negative direction)
for i, elem, s in lhcb1:iter(nil, 2e4, -1) do
	cnt = cnt + i
end
\end{lstlisting}

\begin{figure}[H]
    \centering
    \includegraphics[width=0.8\textwidth]{./Images/seq-iter-curve.pdf}
    \caption{Multiple runs of each cases presented above}
    \label{fig:MO-seq-iter-curve}
\end{figure}

\begin{figure}[H]
    \centering
    \includegraphics[width=0.6\textwidth]{./Images/dump-size}
    \caption{Screenshot of the size of the dump file for those two cases}
    \label{fig:MO-seq-iter}
\end{figure}

\begin{figure}[H]
    \centering
    \includegraphics[width=0.8\textwidth]{./Images/seq-iter-schematic.pdf}
    \caption{Illustration of the \emph{Sequence} in this example}
    \label{fig:MO-seq-iter-schematic}
\end{figure}
