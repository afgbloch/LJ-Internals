%!TEX root = ../FYP_Dissertation.tex

In this project we aimed at studying some surprising performance irregularities
when using LuaJIT in a demanding context. We unrevealed some incompatibility
with the current way LuaJIT was reacted to some structural construct of CERN's
MAD application and provided some insight and potential explanation for its
performance drop. We have done this using the existing tooling provided by
LuaJIT such has the \emph{dump} module and \emph{profiler} module that although
very useful was not sufficient for studying context specific issues on a big code
base. While doing so, we tackled down the problem of the lack of entry-level
documentation that LuaJIT suffers. They do exist some information out there but
it is scattered between a wiki, some Internet post on different websites and most
importantly, sparingly in many mailing-list replies from the author of LuaJIT,
Mike Pall. This work makes a compilation of all this important information
also adding any interesting subtleties discovered while studding LuaJIT internal
implementation.

As for the future of this project, I would say that the next step would be to
invest some amount of time and resources to develop visual tooling to be able
to manipulate and make sense of the JIT reaction and trace dump when facing a
very large number of generated traces. On this subject one shall definitely take
a closer look at the work done by Luke Gorrie on \emph{RaptorJIT}
\cite{RaptorJIT} (a fork of LuaJIT) and more especially on its dedicated
visualization tool \emph{Studio} \cite{Studio}.

I hope this report will help future newcomers to LuaJIT and more especially next
students coming to pick up where we left off and kick-start their Knowledge on
LuaJIT and MAD specificities.
