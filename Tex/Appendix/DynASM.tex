%!TEX root = ../../FYP_Dissertation.tex

DynASM is a Dynamic Assembler for code generation engines, it has been developed
primarily as a tool for LuaJIT and its source code can be found in the dynasm
folder of the LuaJIT project \cite{luajit-src}. It is currently used in LuaJIT
v2 has a tool to write the fast VM interpreting the bytecode. It supports the
following platforms : x86, x64, ARM, PowerPC, and MIPS. An unofficial
documentation with tutorials is available \cite{dynasm}. In the remaining part
of this appendix, a minimum amount of information is presented to give the
ability to read and understand a DynASM file (\emph{.dasc}).


\paratitle{Main directives}

\begin{itemize}
    \item \keyword{.arch} : Specifies the architecture of the assembly code.
    \item \keyword{.type} name, ctype [, default\_reg] : Makes easier to manipulate registers of type \emph{ctype*}. The provided syntactic sugar is depicted in
Table~\ref{tab:type-sugar}.
    \item \keyword{.macro} [...] \keyword{.endmacro} : Create a multi-lines
macro instruction that can be invoked as a normal instruction and where arguments
will be substituted.
    \item \keyword{.define} : Defines a prepreprocessor substitution.
    \item \keyword{.if} [...] \keyword{.elif} [...] \keyword{.else} [...] \keyword{.endif} : Preprocessor conditional construct similar as c preprocessor.
\end{itemize}

\begin{table}
\centering
\begin{tabular}{|l|l|}
\hline
\multicolumn{1}{|c|}{Sugar} & \multicolumn{1}{c|}{Expansion} \\\hline
\#name                      & sizeof(ctype)\\
name:reg-\textgreater field & [reg + offsetof(ctype,field)]\\
name:reg[imm32]             & [reg + sizeof(ctype)*imm32]\\
name:reg[imm32].field       & [reg + sizeof(ctype)*imm32 + offsetof(ctype,field)]\\
name:reg...                 & [reg + (int)(ptrdiff\_t)\&(((ctype*)0)...)]\\\hline
\end{tabular}
\caption{Syntactic sugar provided by \keyword{.type} macro}
\label{tab:type-sugar}
\end{table}

\paratitle{Line markers}\\
Typical DynASM lines that emit some assembler instructions has to start with a
vertical bare ("$\vert$"). If we want to emit some lines of \emph{C} code but
still want DynASM's preprocessor to do substitutions, the line has to start with a double vertical bar ("$\vert\vert$"). Finally lines with no starting markers are
truly untouched and discarded by DynASM. To be noted that lines of \emph{C} code that
has to be inlined with a macro must start with a double vertical bar. \\

\paratitle{Labels}\\
They exist multiple types of labels to refer to. The first category is global
labels that have two types, static label ($\vert-$\textgreater name:) and dynamic ones ($\vert$=\textgreater imm32:).
Those labels have to be unique in a DynASM file. The second category is local
labels that use a single digit from 1 to 9 ($\vert$i:). They can be defined
multiple times inside a same DynASM file and are used by jump instructions by
means of the syntax \textless i or \textgreater i. They respectively point to
the most recent, the next definition of i as the jump target.
