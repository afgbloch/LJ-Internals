% --------------------------------------------------------------------------------
% Student guideline for the final year dissertation using LaTeX2e
% Alexandra Bonnici
% Department of Systems and Control
% 13th September 2017

\documentclass[12pt, oneside]{Thesis}

% --------------------------------------------------------------------------------
% Some useful packages
% You may add/remove any packages as you deem fit,
% Recommended packages which are recommended are marked by [R] in the comment
% --------------------------------------------------------------------------------

% font options & general typesetting
\usepackage{microtype}            % some refinements over the general appearance of text
\usepackage{eqparbox}             % ensures that blocks of text occupy the same space

% for figures
\usepackage[]{graphicx}     %  [R] to use graphics
\usepackage{subfigure}            %  [R] to make subfigures
\usepackage{epstopdf}             %  [R] adds eps support in the graphics package
\usepackage{wrapfig}              %  to allow text to wrap around figures


% for tables
\usepackage[table]{xcolor}
\usepackage{multirow,bigstrut}    % to have tables with merged rows
\usepackage{array, booktabs}      %  [R] for good looking tables
\usepackage{longtable}            % to create tables that span more than one page
\usepackage{adjustbox}            % so that long tables fit within the page
\usepackage{rotating}             % adds support to rotate pages

% to write nice mathematical equations
\usepackage{mathtools}            %  [R] some mathematical tools
\usepackage{amsmath}		      %  [R] to write math formulas easily
\usepackage{amssymb}              %  [R] for maths symbols
\usepackage{amsthm}               % for the typesetting of theorems
\usepackage{esvect}               % to make vector arrows
\usepackage{bm}                   % to make bold maths symbols

\usepackage{xfrac}                % [R] to make in-text fractions

% more symbols
\usepackage{pifont}               % Postscript standard Symbol and Dingbats fonts
\usepackage{gensymb}              % generic symbols fonts
\usepackage{stackrel}             % to stack things on top of each other

% to include urls
\usepackage{url}                  % to write urls

% to typeset pseudo code algorithms and source codes
\usepackage{algorithm}            % to type set algorithms as pseudo code
\usepackage{algpseudocode}        % more layout options for algorithm pseudo code
\usepackage{listings}             % to add non-formatted text e.g. source code

\lstdefinestyle{myLuastyle}
{
  language         = {[5.2]Lua},
  basicstyle       = \ttfamily,
  showstringspaces = false,
  upquote          = true,
}
\lstset{style=myLuastyle}
% to typeset theorems, lemmas and definitions
\newtheorem{theorem}{Theorem}[chapter]
\newtheorem{lemma}{Lemma}
\newtheorem{definition}{Definition}

% \usepackage{enumitem}

% --------------------------------------------------------------------------------
% Information relevant to your dissertation.
% Substitute with your own name, supervisor and title details.
%--------------------------------------------------------------------------------

\thesistitle{LuaJIT Internals \\ {\large v2.1.0-beta3 (1 May 2017)}} % title
\authors{A. \textsc{Bloch} and L. \textsc{Deniau}}	% your name
\keywords{LuaJIT, technical documentation}         	% keywords
%\department{}                						% department name
%\Udegree{}
\subject{LuaJIT Internals}
%\UNIVERSITY{}
%\faculty{Faculty of Engineering}

%\newif\ifHaveACosupervisor
% toggle the comment on the true/false lines as appropriate
%\HaveACosupervisortrue
%\HaveACosupervisorfalse

%--------------------------------------------------------------------------------
% Creates the PDF meta-data (for the electronic version of your dissertation).
% There is no need to modify this information
%--------------------------------------------------------------------------------
\hypersetup{urlcolor=blue,%
		    linkcolor=blue,%
            citecolor=blue,%
            colorlinks=true,%
            hypertexnames=true}

\hypersetup{pdftitle={\ttitle}}
\hypersetup{pdfsubject=\subjectname}
\hypersetup{pdfauthor=\authornames}
\hypersetup{pdfkeywords=\keywordnames}

%--------------------------------------------------------------------------------
% Creates the title page (typesetting is taken care of and details are as above)
%--------------------------------------------------------------------------------
\title{\ttitle}

%--------------------------------------------------------------------------------
% Creates the glossaries pages and index pages
%--------------------------------------------------------------------------------
\makeglossaries
\makeindex

%--------------------------------------------------------------------------------
% The actual start of the dissertation
%--------------------------------------------------------------------------------
\begin{document}

\frontmatter            % sets page numbers of the first few pages to roman
\setstretch{1.3}        % ensures a one-and-a-half line spacing
\fancyhead{}            % clears the headers
\rhead{\thepage}        % sets the top right header to the page number
\lhead{}                % ensures that the top left header remains empty
\pagestyle{fancy}     	% uses the "fancy" page style (adds the line at top)


%--------------------------------------------------------------------------------
% Prints out the title page
%--------------------------------------------------------------------------------
\begin{titlepage}

\flushleft{\textsc{\large European Organization for Nuclear Research -- CERN}\\
\large Beams Department --  Accelerators Beams Physics Group}

\vspace{2cm}
\begin{center}
\linespread{1.3}\huge \bfseries \ttitle
\end{center}
\vspace{2cm}
\begin{center}
by \\[1cm]
\authornames\\[5cm]
\large Dissertation open to the community of LuaJIT \\[1cm]
\vfill
\end{center}

\end{titlepage}


% -------------------------------------------------------------------------------
% Prints out the copyright page (you do not need to modify this)
%--------------------------------------------------------------------------------
\clearpage
\Copyright{

\addtocontents{toc}{\vspace{1em}}
\vspace{2cm}

\begin{enumerate}
\item Copyright in text of this dissertation rests with the Author. Copies (by any process) either in full, or of extracts may be made only in accordance with regulations held by the Library of the University of Malta. Details may be obtained from the Librarian. This page must form part of any such copies made. Further copies (by any process) made in accordance with such instructions may not be made without the permission (in writing) of the Author.
\item Ownership of the right over any original intellectual property which may be contained in or derived from this dissertation is vested in the University of Malta and may not be made available for use by third parties without the written permission of the University, which will prescribe the terms and conditions of any such agreement.
\end{enumerate}}

%--------------------------------------------------------------------------------
% Prints out the abstract
% You will need to change the text in the file to reflect your own abstract
%--------------------------------------------------------------------------------
\clearpage
\addtotoc{Abstract}
\abstract{\addtocontents{toc}{\vspace{1em}}

TODO
}


%--------------------------------------------------------------------------------
% Prints out the acknowledgements page
% You will need to change the text to reflect your own acknowledgements
%--------------------------------------------------------------------------------
\clearpage
\setstretch{1.3}

\acknowledgements{\addtocontents{toc}{\vspace{1em}}
I would like to thank Mike Pall for his commitment for the LuaJIT Project
} 

%--------------------------------------------------------------------------------
% Prints out the list of contents, figures and tables
% You don't have to modify any of this
%--------------------------------------------------------------------------------
\clearpage
\pagestyle{fancy}

\lhead{\emph{Contents}}
\tableofcontents

\lhead{\emph{List of Figures}}
\listoffigures

\lhead{\emph{List of Tables}}
\listoftables

%--------------------------------------------------------------------------------
% Prints out the list of acronyms
% No changes will be necessary if used properly within text
%--------------------------------------------------------------------------------
\clearpage
\setglossarystyle{altlist}
\printglossary[type=\acronymtype, title=List of Acronyms, toctitle=List of Acronyms]

%--------------------------------------------------------------------------------
% Prints out the list of symbols
% No changes will be necessary if used properly within text, comment if not relevant
%--------------------------------------------------------------------------------

\clearpage
\setglossarystyle{list}
\printglossary[title=List of Symbols,toctitle=List of Symbols]


%--------------------------------------------------------------------------------
% Prints out the chapter contents
%--------------------------------------------------------------------------------

\mainmatter
\pagestyle{fancy}

\clearpage
\chapter{Introduction}
\label{Chapt:Introduction}
\lhead{Chapter \thechapter. \emph{Introduction}}
\input{./Chapters/Introduction}

% \clearpage
% \chapter{Overview}
% \label{Chapt:Overview}
% \lhead{Chapter \thechapter. \emph{Overview}}
% %!TEX root = ../FYP_Dissertation.tex
TODO:\\
General introduction on luaJIT (tracing jit).\\
Write a paragraph per Part (Lexer, Parser, VM, JIT).\\
Add a schema on how things are connected(simalar of the one in  Laurent's slide).


\part{Overview}
\label{Part:Overview}

  \chapter{Files hierarchy}
  \label{Chapt:files}
  \lhead{Chapter \thechapter. \emph{Files hierarchy}}
  %!TEX root = ../FYP_Dissertation.tex
\paratitle{dynasm/*:}\\
All files in this folder correspond to the Dynamic Assembler (see Apendix \ref{Apendix:DynASM})\\
\paratitle{dynasm/dasm\_*.h:}\\
DynASM encoding engine for a specific architecture.\\
\paratitle{dynasm/dasm\_*.lua:}\\{}
DynASM lua module for a specific architecture.\\
\paratitle{dynasm/dynasm.lua:}\\
DynASM mechanism main source file.\\
\paratitle{src/host/*:}\\
The files in this directory are only used during the build process of LuaJIT.
For cross-compilation, they must be executed on the host, not on the target.\\
\paratitle{src/host/buildvm.c:}\\
\paratitle{src/host/buildvm.h:}\\
\paratitle{src/host/buildvm\_asm.c:}\\
\paratitle{src/host/buildvm\_fold.c:}\\
\paratitle{src/host/buildvm\_lib.c:}\\
\paratitle{src/host/buildvm\_libbc.h:}\\
\paratitle{src/host/buildvm\_peobj.c:}\\
\paratitle{src/host/genlibbc.lua:}\\
\paratitle{src/host/genminilua.lua:}\\
\paratitle{src/host/minilua.c:}\\
Minimal copy of Lua 5.1 used to build LuaJIT.\\
\paratitle{src/jit/bc.lua:}\\
\paratitle{src/jit/bcsave.lua:}\\
Lua module used to save the bc of a source file.\\
\paratitle{src/jit/dis\_*.lua:}\\
LuaJIT disassembler module for a specific architecture, used as a help module by
the dumper mode to dump code of generated trace (See dump.lua).\\
\paratitle{src/jit/dump.lua:}\\
LuaJIT compiler dump module (see Chapter \ref{Chapt:Dump-mode}).\\
\paratitle{src/jit/p.lua:}\\
LuaJIT profiler (see Chapter \ref{Chapt:Profiler}).\\
\paratitle{src/jit/v.lua:}\\
LuaJIT verbose mode (see Chapter \ref{Chapt:Verbose}).\\
\paratitle{src/jit/zone.lua:}\\
This module implements a simple hierarchical zone model
(See \emph{zone} in Chapter \ref{Chapt:Profiler}).\\
\paratitle{src/lauxlib.h:}\\
Auxiliary library for the Lua/C API.\\
\paratitle{src/lib\_*.c:}\\
Implement a library with an interface available through lua code.\\
\paratitle{src/lib\_aux.c:}\\
\paratitle{src/lib\_aux.c:}\\
\paratitle{src/lib\_base.c:}\\
\paratitle{src/lib\_bit.c:}\\
\paratitle{src/lib\_debug.c:}\\
\paratitle{src/lib\_ffi.c:}\\
FFI library.\\
\paratitle{src/lib\_init.c:}\\
\paratitle{src/lib\_io.c:}\\
\paratitle{src/lib\_jit.c:}\\
\paratitle{src/lib\_math.c:}\\
\paratitle{src/lib\_os.c:}\\
\paratitle{src/lib\_package.c:}\\
\paratitle{src/lib\_string.c:}\\
\paratitle{src/lib\_table.c:}\\
\paratitle{src/lj\_alloc.[ch]:}\\
Bundled memory allocator.\\
\paratitle{src/lj\_api.c:}\\
Lua/c api (e.g. stack handling)\\
\paratitle{src/lj\_arch.h:}\\
\paratitle{src/lj\_asm.[ch]:}\\
Generic code for assembling a trace.\\
\paratitle{src/lj\_asm\_*.h:}\\
IR assembler, convert SSA IR into machine code for a specific architecture.\\
\paratitle{src/lj\_bc.[ch]:}\\
Bytecode instruction format.\\
\paratitle{src/lj\_bcdump.h:}\\
\paratitle{src/lj\_bcread.c:}\\
Bytecode reader to allow executing saved bc.\\
\paratitle{src/lj\_bcwrite.c:}\\
Bytecode writer to save bc.\\
\paratitle{src/lj\_buf.c:}\\
\paratitle{src/lj\_buf.h:}\\
\paratitle{src/lj\_carith.[ch]:}\\
C data arithmetic.\\
\paratitle{src/lj\_ccall.[ch]:}\\
FFI C call handling.\\
\paratitle{src/lj\_ccallback.[ch]:}\\
FFI C callback handling.\\
\paratitle{src/lj\_cconv.[ch]:}\\
C type conversions.\\
\paratitle{src/lj\_cdata.[ch]:}\\
C data management.\\
\paratitle{src/lj\_char.c:}\\
\paratitle{src/lj\_char.h:}\\
\paratitle{src/lj\_clib.[ch]:}\\
FFI C library loader.\\
\paratitle{src/lj\_cparse.[ch]:}\\
C declaration parser.\\
\paratitle{src/lj\_crecord.c:}\\
\paratitle{src/lj\_crecord.h:}\\
\paratitle{src/lj\_ctype.[ch]:}\\
C type management.\\
\paratitle{src/lj\_debug.c:}\\
\paratitle{src/lj\_debug.h:}\\
\paratitle{src/lj\_def.h:}\\
\paratitle{src/lj\_dispatch.c:}\\
\paratitle{src/lj\_dispatch.h:}\\
\paratitle{src/lj\_emit\_*.h:}\\
Instruction emitter for a specific architecture.\\
\paratitle{src/lj\_err.c:}\\
\paratitle{src/lj\_err.h:}\\
\paratitle{src/lj\_errmsg.h:}\\
\paratitle{src/lj\_ff.h:}\\
\paratitle{src/lj\_ffrecord.c:}\\
\paratitle{src/lj\_ffrecord.h:}\\
\paratitle{src/lj\_frame.h:}\\
\paratitle{src/lj\_func.[ch]:}\\
Function handling, prototype and upvalues.\\
\paratitle{src/lj\_gc.[ch]:}\\
Garbage collector implementation\\
\paratitle{src/lj\_gdbjit.c:}\\
\paratitle{src/lj\_gdbjit.h:}\\
\paratitle{src/lj\_ir.c:}\\
\paratitle{src/lj\_ir.h:}\\
\paratitle{src/lj\_ircall.h:}\\
\paratitle{src/lj\_jit.h:}\\
\paratitle{src/lj\_lex.c:}\\
Lexer (Lexical analyzer) for lua code.\\
\paratitle{src/lj\_lex.h:}\\
Main header file used during the lexer, parser and bcreader.\\
\paratitle{src/lj\_lib.c:}\\
\paratitle{src/lj\_lib.h:}\\
\paratitle{src/lj\_load.c:}\\
\paratitle{src/lj\_mcode.c:}\\
\paratitle{src/lj\_mcode.h:}\\
\paratitle{src/lj\_meta.c:}\\
\paratitle{src/lj\_meta.h:}\\
\paratitle{src/lj\_obj.c:}\\
\paratitle{src/lj\_obj.h:}\\
\paratitle{src/lj\_iropt.h:}\\
Common header for IR emitter and optimizations.\\
\paratitle{src/lj\_opt\_*.c:}\\
IR optimization implementation for a specific type of optimization.\\
\paratitle{src/lj\_parse.[ch]:}\\
Lua parser and bc generator\\
\paratitle{src/lj\_profile.c:}\\
\paratitle{src/lj\_profile.h:}\\
\paratitle{src/lj\_record.[ch]:}\\
Trace recorder converts bytecode into SSA IR.\\
\paratitle{src/lj\_snap.[ch]:}\\
Creates and handles trace snapshots.\\
\paratitle{src/lj\_state.c:}\\
\paratitle{src/lj\_state.h:}\\
\paratitle{src/lj\_str.[ch]:}\\
Internal String handling.\\
\paratitle{src/lj\_strfmt.c:}\\
\paratitle{src/lj\_strfmt.h:}\\
\paratitle{src/lj\_strfmt\_num.c:}\\
\paratitle{src/lj\_strscan.c:}\\
\paratitle{src/lj\_strscan.h:}\\
\paratitle{src/lj\_tab.[ch]:}\\
Lua table handling.
\paratitle{src/lj\_target.h:}\\
Generic definitions for target CPU.\\
\paratitle{src/lj\_target\_*.h:}\\
Definitions for target CPU for a specific architecture.\\
\paratitle{src/lj\_trace.[ch]:}\\
Trace management.\\
\paratitle{src/lj\_traceerr.h:}\\
\paratitle{src/lj\_udata.c:}\\
\paratitle{src/lj\_udata.h:}\\
\paratitle{src/lj\_vm.h:}\\
\paratitle{src/lj\_vmevent.c:}\\
\paratitle{src/lj\_vmevent.h:}\\
\paratitle{src/lj\_vmmath.c:}\\
\paratitle{src/ljamalg.c:}\\
LuaJIT core and libraries amalgamation (Used in amalg compilation target to generate better binaries).\\
\paratitle{src/lua.h:}\\
\paratitle{src/lua.hpp:}\\
\paratitle{src/luaconf.h:}\\
\paratitle{src/luajit.c:}\\
\paratitle{src/luajit.h:}\\
\paratitle{src/lualib.h:}\\
\paratitle{src/msvcbuild.bat:}\\
\paratitle{src/ps4build.bat:}\\
\paratitle{src/psvitabuild.bat:}\\
\paratitle{src/vm\_*.dasc:}\\
Virtual Machine written using DynASM for a specific architecture (see Chapter \ref{Chapt:DynASM} for DynASM and Part \ref{Part:VM} for the VM)\\
\paratitle{src/xb1build.bat:}\\
\paratitle{src/xedkbuild.bat:}\\

  \chapter{Major data structure}
  \label{Chapt:ds}

  \chapter{Tools}
  \label{Chapt:Tools}
  \lhead{Chapter \thechapter. \emph{Tools}}
  %!TEX root = ../FYP_Dissertation.tex
\section{Verbose mode}
\label{Sec:Verbose}

This module shows verbose information about the progress of the
JIT compiler. It prints one line for each generated trace. This module
is useful to see which code has been compiled or where the compiler
punts and falls back to the interpreter.

Example usage:

\begin{lstlisting}
  luajit -jv -e "for i=1,1000 do for j=1,1000 do end end"
  luajit -jv=myapp.out myapp.lua
\end{lstlisting}
To redirect the output to a file, pass a
filename as an argument (use '-' for stdout) or set the environment
variable LUAJIT\_VERBOSEFILE. The file is overwritten every time the
module is started.

The output from the second example could look like this:

\begin{center}
[TRACE   1 myapp.lua:1 loop]

[TRACE   2 (1/3) myapp.lua:1 $->$ 1]
\end{center}

The first number in each line is the internal trace number. Next are
the file name ('myapp.lua') and the line number (':1') where the
trace has started. Side traces also show the parent trace number and
the exit number where they are attached to in parentheses ('(1/3)').
An arrow at the end shows where the trace links to ('$->$ 1'), unless
it loops to itself.

In this case the inner loop gets hot and is traced first, generating
a root trace. Then the last exit from the 1st trace gets hot, too,
and triggers generation of the 2nd trace. The side trace follows the
path along the outer loop and \textit{around} the inner loop, back to its
start, and then links to the 1st trace.

Aborted traces are shown like this:
\begin{center}
[TRACE --- foo.lua:44 -- leaving loop in root trace at foo:lua:50]
\end{center}

Trace aborts are quite common, even in programs which
can be fully compiled. The compiler may retry several times until it
finds a suitable trace. This doesn't work with features that are
not-yet-implemented (NYI error messages). The VM simply falls back to the
interpreter. This may not matter at all if the particular trace is not very high
up in the CPU usage profile, plus the interpreter is quite fast, too.

\section{Profiler}
\label{Sec:Profiler}

This module is a simple command line interface to the built-in
low-overhead profiler of LuaJIT. The lower-level API of the profiler
is accessible via the "jit.profile" module or the luaJIT\_profile\_* C API.

Example usage:
\begin{lstlisting}
  luajit -jp myapp.lua
  luajit -jp=s myapp.lua
  luajit -jp=-s myapp.lua
  luajit -jp=vl myapp.lua
  luajit -jp=G,profile.txt myapp.lua
\end{lstlisting}
The following dump features are available:
 \begin{itemize}%[label={--}]
  \item \textbf{f} - Shows function name (Default mode).
  \item \textbf{F} - Shows function name with module prepend.
  \item \textbf{l} - Shows line granularity ('module':'line').
  \item \textbf{\textless number\textgreater} - Stack dump depth (callee\textless caller - Default: 1)
  \item \textbf{-\textless number\textgreater} - Inverse stack dump depth (caller\textgreater callee).
  \item \textbf{s} - Split stack dump after first stack level. Implies $\mid depth\mid$
  \textgreater= 2.
  \item \textbf{p} - Show full path for module names.
  \item \textbf{v} - Show VM states (See bellow).
  \item \textbf{z} - Show zones (See bellow).
  \item \textbf{r} - Show raw sample counts (Default: percentages).
  \item \textbf{a} - Annotate excerpts from source code files.
  \item \textbf{A} - Annotate complete source code files.
  \item \textbf{G} - Produce raw output suitable for graphical tools (See bellow).
  \item \textbf{m\textless number\textgreater} - Minimum sample percentage to be shown (Default: 3).
  \item \textbf{i\textless number\textgreater} - Sampling interval in milliseconds (Default: 10).
 \end{itemize}

 Many of those options can be activated at ones.\\

\paratitle{VM states:}\\
This options allows to shows the time spent in which state of the VM.
States can be of the following types :

\{Compiled - Interpreted - C code - Garbage Collector - JIT Compiler\} \\

\paratitle{Zone:}\\
statistics can be grouped in user defined zone. Bellow is an example a such a
definition.
\begin{lstlisting}
    local zone = require("jit.zone")
    zone("MyZone")
      -- Lua code here
    zone()
\end{lstlisting}

\paratitle{Graphical tools:}\\
This option can be used to graphically show the dump in a nice image format
(see Appendix \ref{Apendix:fl})

\section{Dump mode}
\label{Sec:Dump-mode}

This module can be used to debug the JIT compiler itself. It dumps the
code representations and structures used in various compiler stages.

Example usage:
\begin{lstlisting}
  luajit -jdump -e "
    local x=0
    for i=1,1e6 do x=x+i end
    print(x)
  "
  luajit -jdump=im -e "
    for i=1,1000 do
      for j=1,1000 do end
    end
  " | less -R
  luajit -jdump=is myapp.lua | less -R
  luajit -jdump=-b myapp.lua
  luajit -jdump=+aH,myapp.html myapp.lua
  luajit -jdump=ixT,myapp.dump myapp.lua
\end{lstlisting}
The first argument specifies the dump mode. The second argument gives
the output file name. Default output is to stdout, unless the environment
variable LUAJIT\_DUMPFILE is set. The file is overwritten every time the
module is started. Different features can be turned on or off with the dump mode.
If the mode starts with a '+', the following features are added to the default
set of features; a '-' removes them. Otherwise the features are replaced.\\
The following dump features are available (* marks the default):

\begin{itemize}
  \item \textbf{t *} - Print a line for each started, ended or aborted trace (see also -jv).
  \item \textbf{b *} - Dump the traced bytecode.
  \item \textbf{i *} - Dump the IR (intermediate representation).
  \item \textbf{r} - Augment the IR with register/stack slots.
  \item \textbf{s} - Dump the snapshot map.
  \item \textbf{m *} - Dump the generated machine code.
  \item \textbf{x} - Print each taken trace exit.
  \item \textbf{X} - Print each taken trace exit and the contents of all registers.
  \item \textbf{a} - Print the IR of aborted traces, too.
\end{itemize}
The output format can be set with the following characters:
\begin{itemize}
   \item \textbf{T} - Plain text output.
   \item \textbf{A} - ANSI-colored text output
   \item \textbf{H} - Colorized HTML + CSS output.
\end{itemize}
The default output format is plain text. It's set to ANSI-colored text
if the COLORTERM variable is set.

\part{Interpreter}
\label{Part:Interpreter}

  \chapter{Lexer}
  \label{Chapt:Lexer}

  \chapter{Parser}
  \label{Chapt:Parser}

  \chapter{Bytecode emitter}
  \label{Chapt:BC}

\part{Virtual Machine}
\label{Part:VM}

  \chapter{Bytecode interpreter}
  \label{Chapt:BI}

  \chapter{Foreign function interface}
  \label{Chapt:FFI}

  \chapter{Library function interface}
  \label{Chapt:LFI}

\part{JIT}
\label{Part:JIT}

  \chapter{Trace record}
  \label{Chapt:TR}

  \chapter{Intermediate representation}
  \label{Chapt:IR}

  \chapter{Trace manager}
  \label{Chapt:TM}

  \chapter{Optimizer}
  \label{Chapt:Optimizer}

  \chapter{DynASM: Assembler}
  \label{Chapt:DynASM}
%--------------------------------------------------------------------------------
% Prints out any appendices (comment/delete if you do not need any)
%--------------------------------------------------------------------------------
\clearpage
\addtocontents{toc}{\vspace{2em}}
\appendix
\baselineskip=16pt

\chapter{Flame graphs}
\label{Apendix:fl}
\lhead{Appendix \ref{Apendix:fl}. \emph{Flame graphs}}
%!TEX root = ../FYP_Dissertation.tex
First we need to get the FlameGraph \cite{flamegraph} tool.
\begin{center}
\begin{lstlisting}
  git clone https://github.com/brendangregg/FlameGraph
\end{lstlisting}
\end{center}
Then we need to, generate the raw dump from luajit and use FlameGraph to generate
the svg file. Assuming \emph{\$flamegraph} contains the path to the tool (/FlameGraph/flamegraph.pl)

\begin{center}
\begin{lstlisting}
  export flamegraph=./FlameGraph/flamegraph.pl
  luajit -jp=G,myapp.out myapp.lua
  $flamegraph myapp.out > myapp.svg
\end{lstlisting}
\end{center}
Then you can open myapp.svg with your favorite viewer.


\addtocontents{toc}{\vspace{2em}} % Add a gap in the Contents, for aesthetics

\backmatter

%--------------------------------------------------------------------------------
% Prints out the bibliography
%--------------------------------------------------------------------------------

\label{Bibliography}

\lhead{\emph{Bibliography}}
\renewcommand{\bibname}{References}
\bibliographystyle{MyBibliographyStyle}
\bibliography{./Chapters/FYPBibliography}


\end{document}